\documentclass{amsart}

\input{decls}

%%%%%%
% Draft annotations
\newcommand{\comment}[1]{\textcolor{red}{#1}}
\newcommand{\todo}[1]{\textcolor{red}{#1}}


\title{Semantics of Higher Inductive Types}

\author{Peter LeFanu Lumsdaine}
\author{Michael Shulman}

\begin{document}

\maketitle

\begin{abstract}
The now-standard homotopy-theoretic models of Voevodsky, Awodey--Warren, et al, show that dependent type theory is expressive enough to talk about homotopy-theoretic properties and constructions.
%
However, traditional type theory provides no way to \emph{construct} homotopically non-trivial types.

Higher Inductive Types (HITs) were introduced to remedy this gap (in conjunction with the Univalence Axiom).
%
They generalise ordinary inductive definitions, allowing constructors to produce not only points but \emph{paths} in the posited type.

Here we show that many homotopy-theoretic models of dependent type theory also model HITs, using a generalisation of the initial-algebra semantics of ordinary inductive types.
%
Precisely, we show that \todo{\ldots} .
\end{abstract}


\tableofcontents

\section{Indtrocution}

Some introduction.

\section{Conclusion}

We conclude that the types are high, and they're only getting higher.

\end{document}
