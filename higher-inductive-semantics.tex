\documentclass{amsart}

\input{decls}

%%%%%%
% Draft annotations
\newcommand{\comment}[1]{\textcolor{red}{#1}}
\newcommand{\todo}[1]{\textcolor{red}{#1}}


\title{Semantics of Higher Inductive Types}

\author{Peter LeFanu Lumsdaine}
\author{Michael Shulman}

\begin{document}

\maketitle

\begin{abstract}
The now-standard homotopy-theoretic models of Voevodsky, Awodey--Warren, et al, show that dependent type theory is expressive enough to talk about homotopy-theoretic properties and constructions.
%
However, traditional type theory provides no way to \emph{construct} homotopically non-trivial types.

Higher Inductive Types (HITs) were introduced to remedy this gap (in conjunction with the Univalence Axiom).
%
They generalise ordinary inductive definitions, allowing constructors to produce not only points but \emph{paths} in the posited type.

Here we show that many homotopy-theoretic models of dependent type theory also model HITs, using a generalisation of the initial-algebra semantics of ordinary inductive types.
%
Precisely, we show that \todo{\ldots} .
\end{abstract}


\tableofcontents

\section{Introduction}

\todo{General setting: state of the field, motivation for introducing HITs.}

\todo{Sketch a few examples of HITs}

\todo{Survey work done with HITs in \cite{hott:book}}

\todo{Outline of this paper}

\section{Syntax of HITs}

\todo{Pipe dream: general CIC-like presentation}

\todo{Some simple specific examples: circle, suspension, torus, $(-1)$-truncation \ldots}

\todo{Possible variations, esp.\ judgementality of computation rules}

\todo{Define higher $W$-types, and conjecture their general expressivity}

\section{Semantics of HoTT}

\todo{Recall: comprehension categories, weakly stable structure, \ldots}

\todo{Give specific statement: “Good model categories model HoTT.”}

\todo{While here, define/discuss any extra conditions on model categories we’ll need.}

\todo{Recall: initial-algebra semantics of ITs.}

\section{Semantics of simple HITs}

\todo{Circle: set up cat of ‘algebras’ by hand, note that rules ask exactly for ‘trivially cofibrant’ algebra, in particular initial is enough.}

\todo{Construct initial algebra as pushout.}

\todo{Other simple examples?  Suspension?}

\todo{$(–1)$-truncation: set up the (now less trivial) category of algebras by hand. Note again that an initial one will model the types.}

\todo{Recall \emph{dialgebras}; recognise the algebras above as (iterated) dialgebras.}

\todo{Recall/give initial dialgebra theorems.  Conclude: we have models of the types.}

\section{Semantics of higher $W$-types}

\todo{Set up endofunctors for iterated dialgebra construction.}

\todo{Note that initial dialgebras model higher $W$-types}

\todo{Note that we have initial dialgebras}

\section{Generalisations}

\todo{Higher path constructors, higher recursive calls, weakly natural sources and targets, induction-induction, induction-recursion, etc.}

\todo{How some of these (and simpler examples) can be reduced to higher W-types?}

\todo{Perhaps give or sketch semantics of some higher I–R type?}

\end{document}
