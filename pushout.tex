\documentclass{amsart}

\usepackage{amssymb,amsmath,stmaryrd,mathrsfs}

%% Set this to true before loading if we're using the TAC style file.
%% Note that eventually, TAC requires everything to be in one source file.
\def\definetac{\newif\iftac}    % Can't define a \newif inside another \if!
\ifx\tactrue\undefined
  \definetac
  %% Guess whether we're using TAC by whether \state is defined.
  \ifx\state\undefined\tacfalse\else\tactrue\fi
\fi

% Similarly detect beamer
\def\definebeamer{\newif\ifbeamer}
\ifx\beamertrue\undefined
  \definebeamer
  %% Guess whether we're using Beamer by whether \uncover is defined.
  \ifx\uncover\undefined\beamerfalse\else\beamertrue\fi
\fi

\iftac\else\usepackage{amsthm}\fi
\usepackage[all,2cell]{xy}
%\UseAllTwocells
%\usepackage{tikz}
%\usetikzlibrary{arrows}
\ifbeamer\else
  \usepackage{enumitem}
  \usepackage{xcolor}
  \definecolor{darkgreen}{rgb}{0,0.45,0} 
  \usepackage[pagebackref,colorlinks,citecolor=darkgreen,linkcolor=darkgreen]{hyperref}
\fi
\usepackage{mathtools}          % for all sorts of things
\usepackage{graphics}           % for \scalebox, used in \widecheck
\usepackage{ifmtarg}            % used in \jd
\usepackage{microtype}
%\usepackage{color,epsfig}
%\usepackage{fullpage}
%\usepackage{eucal}
%\usepackage{wasysym}
%\usepackage{txfonts}            % for \invamp, or for the nice fonts
\usepackage{braket}             % for \Set, etc.
\let\setof\Set
\usepackage{url}                % for citations to web sites
\usepackage{xspace}             % put spaces after a \command in text
%\usepackage{cite}               % compress and sort grouped citations (only use with numbered citations)
\usepackage{aliascnt,cleveref}

%% If you want to use biblatex, e.g. if a journal requires (Author name YEAR) citations.
% \usepackage[style=authoryear,
%  backref=true,
%  maxnames=4,
%  maxbibnames=99,
%  uniquename=false,
%  firstinits=true
% ]{biblatex}
% \addbibresource{all.bib}

% \let\cite\parencite
% \DeclareNameAlias{sortname}{last-first}

\makeatletter
\let\ea\expandafter

%% Defining commands that are always in math mode.
\def\mdef#1#2{\ea\ea\ea\gdef\ea\ea\noexpand#1\ea{\ea\ensuremath\ea{#2}\xspace}}
\def\alwaysmath#1{\ea\ea\ea\global\ea\ea\ea\let\ea\ea\csname your@#1\endcsname\csname #1\endcsname
  \ea\def\csname #1\endcsname{\ensuremath{\csname your@#1\endcsname}\xspace}}

%% WIDECHECK
\DeclareRobustCommand\widecheck[1]{{\mathpalette\@widecheck{#1}}}
\def\@widecheck#1#2{%
    \setbox\z@\hbox{\m@th$#1#2$}%
    \setbox\tw@\hbox{\m@th$#1%
       \widehat{%
          \vrule\@width\z@\@height\ht\z@
          \vrule\@height\z@\@width\wd\z@}$}%
    \dp\tw@-\ht\z@
    \@tempdima\ht\z@ \advance\@tempdima2\ht\tw@ \divide\@tempdima\thr@@
    \setbox\tw@\hbox{%
       \raise\@tempdima\hbox{\scalebox{1}[-1]{\lower\@tempdima\box
\tw@}}}%
    {\ooalign{\box\tw@ \cr \box\z@}}}

%% SIMPLE COMMANDS FOR FONTS AND DECORATIONS

\newcount\foreachcount

\def\foreachletter#1#2#3{\foreachcount=#1
  \ea\loop\ea\ea\ea#3\@alph\foreachcount
  \advance\foreachcount by 1
  \ifnum\foreachcount<#2\repeat}

\def\foreachLetter#1#2#3{\foreachcount=#1
  \ea\loop\ea\ea\ea#3\@Alph\foreachcount
  \advance\foreachcount by 1
  \ifnum\foreachcount<#2\repeat}

% Script: \sA is \mathscr{A}
\def\definescr#1{\ea\gdef\csname s#1\endcsname{\ensuremath{\mathscr{#1}}\xspace}}
\foreachLetter{1}{27}{\definescr}
% Calligraphic: \cA is \mathcal{A}
\def\definecal#1{\ea\gdef\csname c#1\endcsname{\ensuremath{\mathcal{#1}}\xspace}}
\foreachLetter{1}{27}{\definecal}
% Bold: \bA is \mathbf{A}
\def\definebold#1{\ea\gdef\csname b#1\endcsname{\ensuremath{\mathbf{#1}}\xspace}}
\foreachLetter{1}{27}{\definebold}
% Blackboard Bold: \lA is \mathbb{A}
\def\definebb#1{\ea\gdef\csname l#1\endcsname{\ensuremath{\mathbb{#1}}\xspace}}
\foreachLetter{1}{27}{\definebb}
% Fraktur: \ka is \mathfrak{a} (except when it's \kappa, see below), \kA is \mathfrak{A}
\def\definefrak#1{\ea\gdef\csname k#1\endcsname{\ensuremath{\mathfrak{#1}}\xspace}}
\foreachletter{1}{27}{\definefrak}
\foreachLetter{1}{27}{\definefrak}
% Sans serif
\def\definesf#1{\ea\gdef\csname i#1\endcsname{\ensuremath{\mathsf{#1}}\xspace}}
\foreachletter{1}{6}{\definesf}
\foreachletter{7}{14}{\definesf}
\foreachletter{15}{27}{\definesf}
\foreachLetter{1}{27}{\definesf}
% Bar: \Abar is \overline{A}, \abar is \overline{a}
\def\definebar#1{\ea\gdef\csname #1bar\endcsname{\ensuremath{\overline{#1}}\xspace}}
\foreachLetter{1}{27}{\definebar}
\foreachletter{1}{8}{\definebar} % \hbar is something else!
\foreachletter{9}{15}{\definebar} % \obar is something else!
\foreachletter{16}{27}{\definebar}
% Tilde: \Atil is \widetilde{A}, \atil is \widetilde{a}
\def\definetil#1{\ea\gdef\csname #1til\endcsname{\ensuremath{\widetilde{#1}}\xspace}}
\foreachLetter{1}{27}{\definetil}
\foreachletter{1}{27}{\definetil}
% Hats: \Ahat is \widehat{A}, \ahat is \widehat{a}
\def\definehat#1{\ea\gdef\csname #1hat\endcsname{\ensuremath{\widehat{#1}}\xspace}}
\foreachLetter{1}{27}{\definehat}
\foreachletter{1}{27}{\definehat}
% Checks: \Achk is \widecheck{A}, \achk is \widecheck{a}
\def\definechk#1{\ea\gdef\csname #1chk\endcsname{\ensuremath{\widecheck{#1}}\xspace}}
\foreachLetter{1}{27}{\definechk}
\foreachletter{1}{27}{\definechk}
% Underline: \uA is \underline{A}, \ua is \underline{a}
\def\defineul#1{\ea\gdef\csname u#1\endcsname{\ensuremath{\underline{#1}}\xspace}}
\foreachLetter{1}{27}{\defineul}
\foreachletter{1}{27}{\defineul}

% Particular commands for typefaces, sometimes with the first letter
% different.
\def\autofmt@n#1\autofmt@end{\mathrm{#1}}
\def\autofmt@b#1\autofmt@end{\mathbf{#1}}
\def\autofmt@l#1#2\autofmt@end{\mathbb{#1}\mathsf{#2}}
\def\autofmt@c#1#2\autofmt@end{\mathcal{#1}\mathit{#2}}
\def\autofmt@s#1#2\autofmt@end{\mathscr{#1}\mathit{#2}}
\def\autofmt@f#1\autofmt@end{\mathsf{#1}}
\def\autofmt@k#1\autofmt@end{\mathfrak{#1}}
% Particular commands for decorations.
\def\autofmt@u#1\autofmt@end{\underline{\smash{\mathsf{#1}}}}
\def\autofmt@U#1\autofmt@end{\underline{\underline{\smash{\mathsf{#1}}}}}
\def\autofmt@h#1\autofmt@end{\widehat{#1}}
\def\autofmt@r#1\autofmt@end{\overline{#1}}
\def\autofmt@t#1\autofmt@end{\widetilde{#1}}
\def\autofmt@k#1\autofmt@end{\check{#1}}

% Defining multi-letter commands.  Use this like so:
% \autodefs{\bSet\cCat\cCAT\kBicat\lProf}
\def\auto@drop#1{}
\def\autodef#1{\ea\ea\ea\@autodef\ea\ea\ea#1\ea\auto@drop\string#1\autodef@end}
\def\@autodef#1#2#3\autodef@end{%
  \ea\def\ea#1\ea{\ea\ensuremath\ea{\csname autofmt@#2\endcsname#3\autofmt@end}\xspace}}
\def\autodefs@end{blarg!}
\def\autodefs#1{\@autodefs#1\autodefs@end}
\def\@autodefs#1{\ifx#1\autodefs@end%
  \def\autodefs@next{}%
  \else%
  \def\autodefs@next{\autodef#1\@autodefs}%
  \fi\autodefs@next}

%% FONTS AND DECORATION FOR GREEK LETTERS

%% the package `mathbbol' gives us blackboard bold greek and numbers,
%% but it does it by redefining \mathbb to use a different font, so that
%% all the other \mathbb letters look different too.  Here we import the
%% font with bb greek and numbers, but assign it a different name,
%% \mathbbb, so as not to replace the usual one.
\DeclareSymbolFont{bbold}{U}{bbold}{m}{n}
\DeclareSymbolFontAlphabet{\mathbbb}{bbold}
\newcommand{\lDelta}{\ensuremath{\mathbbb{\Delta}}\xspace}
\newcommand{\lone}{\ensuremath{\mathbbb{1}}\xspace}
\newcommand{\ltwo}{\ensuremath{\mathbbb{2}}\xspace}
\newcommand{\lthree}{\ensuremath{\mathbbb{3}}\xspace}

% greek with bars
\newcommand{\albar}{\ensuremath{\overline{\alpha}}\xspace}
\newcommand{\bebar}{\ensuremath{\overline{\beta}}\xspace}
\newcommand{\gmbar}{\ensuremath{\overline{\gamma}}\xspace}
\newcommand{\debar}{\ensuremath{\overline{\delta}}\xspace}
\newcommand{\phibar}{\ensuremath{\overline{\varphi}}\xspace}
\newcommand{\psibar}{\ensuremath{\overline{\psi}}\xspace}
\newcommand{\xibar}{\ensuremath{\overline{\xi}}\xspace}
\newcommand{\ombar}{\ensuremath{\overline{\omega}}\xspace}

% greek with tildes
\newcommand{\altil}{\ensuremath{\widetilde{\alpha}}\xspace}
\newcommand{\betil}{\ensuremath{\widetilde{\beta}}\xspace}
\newcommand{\gmtil}{\ensuremath{\widetilde{\gamma}}\xspace}
\newcommand{\phitil}{\ensuremath{\widetilde{\varphi}}\xspace}
\newcommand{\psitil}{\ensuremath{\widetilde{\psi}}\xspace}
\newcommand{\xitil}{\ensuremath{\widetilde{\xi}}\xspace}
\newcommand{\omtil}{\ensuremath{\widetilde{\omega}}\xspace}

% MISCELLANEOUS SYMBOLS
\let\del\partial
\mdef\delbar{\overline{\partial}}
\let\sm\wedge
\newcommand{\dd}[1]{\ensuremath{\frac{\partial}{\partial {#1}}}}
\newcommand{\inv}{^{-1}}
\newcommand{\dual}{^{\vee}}
\mdef\hf{\textstyle\frac12 }
\mdef\thrd{\textstyle\frac13 }
\mdef\qtr{\textstyle\frac14 }
\let\meet\wedge
\let\join\vee
\let\dn\downarrow
\newcommand{\op}{^{\mathrm{op}}}
\newcommand{\co}{^{\mathrm{co}}}
\newcommand{\coop}{^{\mathrm{coop}}}
\let\adj\dashv
\SelectTips{cm}{}
\newdir{ >}{{}*!/-10pt/\dir{>}}    % extra spacing for tail arrows in XYpic
\newcommand{\pushout}[1][dr]{\save*!/#1+1.2pc/#1:(1,-1)@^{|-}\restore}
\newcommand{\pullback}[1][dr]{\save*!/#1-1.2pc/#1:(-1,1)@^{|-}\restore}
\let\iso\cong
\let\eqv\simeq
\let\cng\equiv
% \mdef\Id{\mathrm{Id}}
% \mdef\id{\mathrm{id}}
\alwaysmath{ell}
\alwaysmath{infty}
\alwaysmath{odot}
\def\frc#1/#2.{\frac{#1}{#2}}   % \frc x^2+1 / x^2-1 .
\mdef\ten{\mathrel{\otimes}}
\let\bigten\bigotimes
\mdef\sqten{\mathrel{\boxtimes}}
\def\lt{<}                      % For iTex compatibility
\def\gt{>}

%% OPERATORS
\DeclareMathOperator\lan{Lan}
\DeclareMathOperator\ran{Ran}
\DeclareMathOperator\colim{colim}
\DeclareMathOperator\coeq{coeq}
\DeclareMathOperator\eq{eq}
\DeclareMathOperator\Tot{Tot}
\DeclareMathOperator\cosk{cosk}
\DeclareMathOperator\sk{sk}
%\DeclareMathOperator\im{im}
\DeclareMathOperator\Spec{Spec}
\DeclareMathOperator\Ho{Ho}
\DeclareMathOperator\Aut{Aut}
\DeclareMathOperator\End{End}
\DeclareMathOperator\Hom{Hom}
\DeclareMathOperator\Map{Map}

%% ARROWS
% \to already exists
\newcommand{\too}[1][]{\ensuremath{\overset{#1}{\longrightarrow}}}
\newcommand{\ot}{\ensuremath{\leftarrow}}
\newcommand{\oot}[1][]{\ensuremath{\overset{#1}{\longleftarrow}}}
\let\toot\rightleftarrows
\let\otto\leftrightarrows
\let\Impl\Rightarrow
\let\imp\Rightarrow
\let\toto\rightrightarrows
\let\into\hookrightarrow
\let\xinto\xhookrightarrow
\mdef\we{\overset{\sim}{\longrightarrow}}
\mdef\leftwe{\overset{\sim}{\longleftarrow}}
\let\mono\rightarrowtail
\let\leftmono\leftarrowtail
\let\cof\rightarrowtail
\let\leftcof\leftarrowtail
\let\epi\twoheadrightarrow
\let\leftepi\twoheadleftarrow
\let\fib\twoheadrightarrow
\let\leftfib\twoheadleftarrow
\let\cohto\rightsquigarrow
\let\maps\colon
\newcommand{\spam}{\,:\!}       % \maps for left arrows
\def\acof{\mathrel{\mathrlap{\hspace{3pt}\raisebox{4pt}{$\scriptscriptstyle\sim$}}\mathord{\rightarrowtail}}}

% diagxy redefines \to, along with \toleft, \two, \epi, and \mon.

%% EXTENSIBLE ARROWS
\let\xto\xrightarrow
\let\xot\xleftarrow
% See Voss' Mathmode.tex for instructions on how to create new
% extensible arrows.
\def\rightarrowtailfill@{\arrowfill@{\Yright\joinrel\relbar}\relbar\rightarrow}
\newcommand\xrightarrowtail[2][]{\ext@arrow 0055{\rightarrowtailfill@}{#1}{#2}}
\let\xmono\xrightarrowtail
\let\xcof\xrightarrowtail
\def\twoheadrightarrowfill@{\arrowfill@{\relbar\joinrel\relbar}\relbar\twoheadrightarrow}
\newcommand\xtwoheadrightarrow[2][]{\ext@arrow 0055{\twoheadrightarrowfill@}{#1}{#2}}
\let\xepi\xtwoheadrightarrow
\let\xfib\xtwoheadrightarrow
% Let's leave the left-going ones until I need them.

%% EXTENSIBLE SLASHED ARROWS
% Making extensible slashed arrows, by modifying the underlying AMS code.
% Arguments are:
% 1 = arrowhead on the left (\relbar or \Relbar if none)
% 2 = fill character (usually \relbar or \Relbar)
% 3 = slash character (such as \mapstochar or \Mapstochar)
% 4 = arrowhead on the left (\relbar or \Relbar if none)
% 5 = display mode (\displaystyle etc)
\def\slashedarrowfill@#1#2#3#4#5{%
  $\m@th\thickmuskip0mu\medmuskip\thickmuskip\thinmuskip\thickmuskip
   \relax#5#1\mkern-7mu%
   \cleaders\hbox{$#5\mkern-2mu#2\mkern-2mu$}\hfill
   \mathclap{#3}\mathclap{#2}%
   \cleaders\hbox{$#5\mkern-2mu#2\mkern-2mu$}\hfill
   \mkern-7mu#4$%
}
% Here's the idea: \<slashed>arrowfill@ should be a box containing
% some stretchable space that is the "middle of the arrow".  This
% space is created as a "leader" using \cleader<thing>\hfill, which
% fills an \hfill of space with copies of <thing>.  Here instead of
% just one \cleader, we use two, with the slash in between (and an
% extra copy of the filler, to avoid extra space around the slash).
\def\rightslashedarrowfill@{%
  \slashedarrowfill@\relbar\relbar\mapstochar\rightarrow}
\newcommand\xslashedrightarrow[2][]{%
  \ext@arrow 0055{\rightslashedarrowfill@}{#1}{#2}}
\mdef\hto{\xslashedrightarrow{}}
\mdef\htoo{\xslashedrightarrow{\quad}}
\let\xhto\xslashedrightarrow

%% To get a slashed arrow in XYmatrix, do
% \[\xymatrix{A \ar[r]|-@{|} & B}\]
%% To get it in diagxy, do
% \morphism/{@{>}|-*@{|}}/[A`B;p]

%% Here is an \hto for diagxy:
% \def\htopppp/#1/<#2>^#3_#4{\:%
% \ifnum#2=0%
%    \setwdth{#3}{#4}\deltax=\wdth \divide \deltax by \ul%
%    \advance \deltax by \defaultmargin  \ratchet{\deltax}{100}%
% \else \deltax #2%
% \fi%
% \xy\ar@{#1}|-@{|}^{#3}_{#4}(\deltax,0) \endxy%
% \:}%
% \def\htoppp/#1/<#2>^#3{\ifnextchar_{\htopppp/#1/<#2>^{#3}}{\htopppp/#1/<#2>^{#3}_{}}}%
% \def\htopp/#1/<#2>{\ifnextchar^{\htoppp/#1/<#2>}{\htoppp/#1/<#2>^{}}}%
% \def\htoop/#1/{\ifnextchar<{\htopp/#1/}{\htopp/#1/<0>}}%
% \def\hto{\ifnextchar/{\htoop}{\htoop/>/}}%

% LABELED ISOMORPHISMS
\def\xiso#1{\mathrel{\mathrlap{\smash{\xto[\smash{\raisebox{1.3mm}{$\scriptstyle\sim$}}]{#1}}}\hphantom{\xto{#1}}}}
\def\toiso{\xto{\smash{\raisebox{-.5mm}{$\scriptstyle\sim$}}}}

% SHADOWS
\def\shvar#1#2{{\ensuremath{%
  \hspace{1mm}\makebox[-1mm]{$#1\langle$}\makebox[0mm]{$#1\langle$}\hspace{1mm}%
  {#2}%
  \makebox[1mm]{$#1\rangle$}\makebox[0mm]{$#1\rangle$}%
}}}
\def\sh{\shvar{}}
\def\scriptsh{\shvar{\scriptstyle}}
\def\bigsh{\shvar{\big}}
\def\Bigsh{\shvar{\Big}}
\def\biggsh{\shvar{\bigg}}
\def\Biggsh{\shvar{\Bigg}}

% % TYPING JUDGMENTS
% % Call this macro as \jd{x:A, y:B |- c:C}.  It adds (what I think is)
% % appropriate spacing, plus auto-sized parentheses around each typing judgment.
% \def\jd#1{\@jd#1\ej}
% \def\@jd#1|-#2\ej{\@@jd#1,,\;\vdash\;#2}
% \def\@@jd#1,{\@ifmtarg{#1}{\let\next=\relax}{\left(#1\right)\let\next=\@@@jd}\next}
% \def\@@@jd#1,{\@ifmtarg{#1}{\let\next=\relax}{,\,\left(#1\right)\let\next=\@@@jd}\next}
% % Here's a version which puts a line break before the turnstyle.
% \def\jdm#1{\@jdm#1\ej}
% \def\@jdm#1|-#2\ej{\@@jd#1,,\\\vdash\;\left(#2\right)}
% % Make an actual comma that doesn't separate typing judgments (e.g. A,B,C : Type).
% \def\cm{,}

%% SKIPIT in TikZ
% See http://tex.stackexchange.com/questions/3513/draw-only-some-segments-of-a-path-in-tikz
\long\def\my@drawfill#1#2;{%
\@skipfalse
\fill[#1,draw=none] #2;
\@skiptrue
\draw[#1,fill=none] #2;
}
\newif\if@skip
\newcommand{\skipit}[1]{\if@skip\else#1\fi}
\newcommand{\drawfill}[1][]{\my@drawfill{#1}}

% How to get QED symbols inside equations at the end of the statements
% of theorems.  AMS LaTeX knows how to do this inside equations at the
% end of *proofs* with \qedhere, and at the end of the statement of a
% theorem with \qed (meaning no proof will be given), but it can't
% seem to combine the two.  Use this instead.
\def\thmqedhere{\expandafter\csname\csname @currenvir\endcsname @qed\endcsname}

% Number numbered lists as (i), (ii), ...
\ifbeamer\else
  \renewcommand{\theenumi}{(\roman{enumi})}
  \renewcommand{\labelenumi}{\theenumi}
\fi

% Left margins for enumitem
\ifbeamer\else
  \setitemize[1]{leftmargin=2em}
  \setenumerate[1]{leftmargin=*}
\fi

% Only show numbers for equations that are actually referenced (or
% whose tags are specified manually) - uses mathtools.  All equations
% need to be referenced with \eqref, not \ref, for this to work!
%\@ifpackageloaded{mathtools}{\mathtoolsset{showonlyrefs,showmanualtags}}{}

% GREEK LETTERS, ETC.
\alwaysmath{alpha}
\alwaysmath{beta}
\alwaysmath{gamma}
\alwaysmath{Gamma}
\alwaysmath{delta}
\alwaysmath{Delta}
\alwaysmath{epsilon}
\mdef\ep{\varepsilon}
\alwaysmath{zeta}
\alwaysmath{eta}
\alwaysmath{theta}
\alwaysmath{Theta}
\alwaysmath{iota}
\alwaysmath{kappa}
\alwaysmath{lambda}
\alwaysmath{Lambda}
\alwaysmath{mu}
\alwaysmath{nu}
\alwaysmath{xi}
\alwaysmath{pi}
\alwaysmath{rho}
\alwaysmath{sigma}
\alwaysmath{Sigma}
\alwaysmath{tau}
\alwaysmath{upsilon}
\alwaysmath{Upsilon}
\alwaysmath{phi}
\alwaysmath{Pi}
\alwaysmath{Phi}
\mdef\ph{\varphi}
\alwaysmath{chi}
\alwaysmath{psi}
\alwaysmath{Psi}
\alwaysmath{omega}
\alwaysmath{Omega}
\let\al\alpha
\let\be\beta
\let\gm\gamma
\let\Gm\Gamma
\let\de\delta
\let\De\Delta
\let\si\sigma
\let\Si\Sigma
\let\om\omega
\let\ka\kappa
\let\la\lambda
\let\La\Lambda
\let\ze\zeta
\let\th\theta
\let\Th\Theta
\let\vth\vartheta
\let\Om\Omega

%% Include or exclude solutions
% This code is basically copied from version.sty, except that when the
% solutions are included, we put them in a `proof' environment as
% well.  To include solutions, say \includesolutions; to exclude them
% say \excludesolutions.
% \begingroup
% 
% \catcode`{=12\relax\catcode`}=12\relax%
% \catcode`(=1\relax \catcode`)=2\relax%
% \gdef\includesolutions(\newenvironment(soln)(\begin(proof)[Solution])(\end(proof)))%
% \gdef\excludesolutions(%
%   \gdef\soln(\@bsphack\catcode`{=12\relax\catcode`}=12\relax\soln@NOTE)%
%   \long\gdef\soln@NOTE##1\end{soln}(\solnEND@NOTE)%
%   \gdef\solnEND@NOTE(\@esphack\end(soln))%
% )%
% \endgroup

\makeatother

% Local Variables:
% mode: latex
% TeX-master: "higher-inductive-semantics"
% End:

%%%%%%
% Draft annotations

\usepackage[mode=multiuser,layout=margin,status=draft]{fixme}
\FXRegisterAuthor{ms}{ams}{MS}  % Use \msnote, \mswarning, \mserror, \msfatal
\FXRegisterAuthor{pl}{apl}{PLL} % Use \plnote, \plwarning, \plerror, \plfatal
\fxusetheme{color}

%\newcommand{\comment}[1]{\textcolor{red}{#1}}
\newcommand{\todo}[1]{\textcolor{red}{#1}}

%%%%%%
% Theorem-style environments

\theoremstyle{theorem}
\newtheorem{theorem}{Theorem}[section]
\theoremstyle{definition}
\newtheorem{definition}[theorem]{Definition}

%%%%%
% Text in typefaces

\autodefs{\fap\fcomp\fzero\fsucc\fpair\ftt\fbase\floop\fone\fseg\fnorth\fsouth\fmerid\fsurf\fext\fdata\fwhere\fId\fEq\fproj\fsquash\frefl\fType\ftype\ffib\fVec\fnil\fcons\fdim\fapeq\fcoeq}

\let\E\cE
\let\refl\frefl

%%%%%
% Inductive definitions

\usepackage{xifthen}
\makeatletter
\def\indeff#1#2#3{
  \begin{quote}
    \noindent \fdata ${#1} : {#2}$ \fwhere
    \@indef #3 \OR\OR
  \end{quote}
}
\def\indef#1#2{\indeff{#1}{\fType}{#2}}
\def\@indef#1\OR{\ifthenelse{\isempty{#1}}{}{\\\hspace*{3mm} $#1$ \@indef}}

%%%%%
% General macros

% Judgments
\usepackage{mathpartir}
\newcommand{\type}{\;\ftype}
\renewcommand{\fib}{\;\ffib}
\renewcommand{\dim}{\;\fdim}
\newcommand{\subst}[2]{[#1/#2]}
\newcommand{\cb}{\,;\,}         % context break
\newcommand{\pr}{\,\vdash\,}
\newcommand{\emptycxt}{\diamond}

% Dimension variables
\newcommand{\dsubst}[2]{\langle #1/#2\rangle}
\newcommand{\dlamsym}{\mathring{\lambda}}
\newcommand{\dlam}[1]{\dlamsym#1.\,}
\newcommand{\dapp}{@}

% Induction
\newcommand{\ind}[1]{\mathsf{ind}_{#1}}

% Categorical/logical constructions
\renewcommand{\id}[3][]{\fId_{#1}(#2,#3)}
\newcommand{\idover}[4][]{\fId_{#1}(#2,#3)_{#4}}
\newcommand{\paths}[2][]{\cP_{#1}(#2)}
\newcommand{\pathsover}[3][]{\cP_{#1}(#3;#2)}
\let\ap\fap
\newcommand{\circtype}{\ensuremath{S^1}\xspace}
\newcommand{\torustype}{\ensuremath{T^2}\xspace}
\newcommand{\spheretype}[1]{\ensuremath{S^{#1}}\xspace}
\newcommand{\unittype}{\ensuremath{\mathbf{1}}\xspace}
\newcommand{\trunc}[2]{\mathopen{}\left\Vert #2\right\Vert_{#1}\mathclose{}}
\newcommand{\brck}[1]{\trunc{}{#1}}
\newcommand{\freegroup}[1]{F #1}
\newcommand{\local}[2]{L_{#1} #2}
\newcommand{\susp}{\Sigma}

% Strict things
\renewcommand{\eq}[3][]{\fEq_{#1}(#2,#3)}
\renewcommand{\cof}[3]{#1 : #2 \rightarrowtail #3}
\newcommand{\pop}{\Box}         % pushout product

% Categories
\newcommand{\Alg}[2][]{{#2}\text{-}\mathbf{Alg}_{#1}}
\newcommand{\Eself}{\E/\blank}  % self-indexing

% Functors and monads
\newcommand{\freemonad}[1]{\overline{#1}}

% Arrows
\newcommand{\fibto}{\to}

%%%%%
% Borrowed from the HoTT Book

% Judgmental equality
\newcommand{\jdeq}{\equiv}

% Placeholders
\newcommand{\blank}{\mathord{\hspace{1pt}\text{--}\hspace{1pt}}}

% Path reversal
\newcommand{\opp}[1]{\mathord{{#1}^{-1}}}

% Transport
\newcommand{\transf}[1]{\ensuremath{{#1}_{*}}\xspace} % Without argument

%%% Path concatenation (used infix, in diagrammatic order) %%%
\newcommand{\ct}{%
  \mathchoice{\mathbin{\raisebox{0.5ex}{$\displaystyle\centerdot$}}}%
             {\mathbin{\raisebox{0.5ex}{$\centerdot$}}}%
             {\mathbin{\raisebox{0.25ex}{$\scriptstyle\,\centerdot\,$}}}%
             {\mathbin{\raisebox{0.1ex}{$\scriptscriptstyle\,\centerdot\,$}}}
}

% Dependent products
\def\prdsym{\textstyle\prod}
%% Call the macro like \prd{x,y:A}{p:x=y} with any number of
%% arguments.  Make sure that whatever comes *after* the call doesn't
%% begin with an open-brace, or it will be parsed as another argument.
\makeatletter
% Currently the macro is configured to produce
%     {\textstyle\prod}(x:A) \; {\textstyle\prod}(y:B),{\ }
% in display-math mode, and
%     \prod_{(x:A)} \prod_{y:B}
% in text-math mode.
% \def\prd#1{\@ifnextchar\bgroup{\prd@parens{#1}}{%
%     \@ifnextchar\sm{\prd@parens{#1}\@eatsm}{%
%         \prd@noparens{#1}}}}
\def\prd#1{\@ifnextchar\bgroup{\prd@parens{#1}}{%
    \@ifnextchar\sm{\prd@parens{#1}\@eatsm}{%
    \@ifnextchar\prd{\prd@parens{#1}\@eatprd}{%
    \@ifnextchar\;{\prd@parens{#1}\@eatsemicolonspace}{%
    \@ifnextchar\\{\prd@parens{#1}\@eatlinebreak}{%
    \@ifnextchar\narrowbreak{\prd@parens{#1}\@eatnarrowbreak}{%
      \prd@noparens{#1}}}}}}}}
\def\prd@parens#1{\@ifnextchar\bgroup%
  {\mathchoice{\@dprd{#1}}{\@tprd{#1}}{\@tprd{#1}}{\@tprd{#1}}\prd@parens}%
  {\@ifnextchar\sm%
    {\mathchoice{\@dprd{#1}}{\@tprd{#1}}{\@tprd{#1}}{\@tprd{#1}}\@eatsm}%
    {\mathchoice{\@dprd{#1}}{\@tprd{#1}}{\@tprd{#1}}{\@tprd{#1}}}}}
\def\@eatsm\sm{\sm@parens}
\def\prd@noparens#1{\mathchoice{\@dprd@noparens{#1}}{\@tprd{#1}}{\@tprd{#1}}{\@tprd{#1}}}
% Helper macros for three styles
\def\lprd#1{\@ifnextchar\bgroup{\@lprd{#1}\lprd}{\@@lprd{#1}}}
\def\@lprd#1{\mathchoice{{\textstyle\prod}}{\prod}{\prod}{\prod}({\textstyle #1})\;}
\def\@@lprd#1{\mathchoice{{\textstyle\prod}}{\prod}{\prod}{\prod}({\textstyle #1}),\ }
\def\tprd#1{\@tprd{#1}\@ifnextchar\bgroup{\tprd}{}}
\def\@tprd#1{\mathchoice{{\textstyle\prod_{(#1)}}}{\prod_{(#1)}}{\prod_{(#1)}}{\prod_{(#1)}}}
\def\dprd#1{\@dprd{#1}\@ifnextchar\bgroup{\dprd}{}}
\def\@dprd#1{\prod_{(#1)}\,}
\def\@dprd@noparens#1{\prod_{#1}\,}

% Look through spaces and linebreaks
\def\@eatnarrowbreak\narrowbreak{%
  \@ifnextchar\prd{\narrowbreak\@eatprd}{%
    \@ifnextchar\sm{\narrowbreak\@eatsm}{%
      \narrowbreak}}}
\def\@eatlinebreak\\{%
  \@ifnextchar\prd{\\\@eatprd}{%
    \@ifnextchar\sm{\\\@eatsm}{%
      \\}}}
\def\@eatsemicolonspace\;{%
  \@ifnextchar\prd{\;\@eatprd}{%
    \@ifnextchar\sm{\;\@eatsm}{%
      \;}}}

%%% Lambda abstractions.
% Each variable being abstracted over is a separate argument.  If
% there is more than one such argument, they *must* be enclosed in
% braces.  Arguments can be untyped, as in \lam{x}{y}, or typed with a
% colon, as in \lam{x:A}{y:B}. In the latter case, the colons are
% automatically noticed and (with current implementation) the space
% around the colon is reduced.  You can even give more than one variable
% the same type, as in \lam{x,y:A}.
\def\lam#1{{\lambda}\@lamarg#1:\@endlamarg\@ifnextchar\bgroup{.\,\lam}{.\,}}
\def\@lamarg#1:#2\@endlamarg{\if\relax\detokenize{#2}\relax #1\else\@lamvar{\@lameatcolon#2},#1\@endlamvar\fi}
\def\@lamvar#1,#2\@endlamvar{(#2\,{:}\,#1)}
% \def\@lamvar#1,#2{{#2}^{#1}\@ifnextchar,{.\,{\lambda}\@lamvar{#1}}{\let\@endlamvar\relax}}
\def\@lameatcolon#1:{#1}
\let\lamt\lam
% This version silently eats any typing annotation.
\def\lamu#1{{\lambda}\@lamuarg#1:\@endlamuarg\@ifnextchar\bgroup{.\,\lamu}{.\,}}
\def\@lamuarg#1:#2\@endlamuarg{#1}


%%% Dependent sums %%%
\def\smsym{\textstyle\sum}
% Use in the same way as \prd
\def\sm#1{\@ifnextchar\bgroup{\sm@parens{#1}}{%
    \@ifnextchar\prd{\sm@parens{#1}\@eatprd}{%
    \@ifnextchar\sm{\sm@parens{#1}\@eatsm}{%
    \@ifnextchar\;{\sm@parens{#1}\@eatsemicolonspace}{%
    \@ifnextchar\\{\sm@parens{#1}\@eatlinebreak}{%
    \@ifnextchar\narrowbreak{\sm@parens{#1}\@eatnarrowbreak}{%
        \sm@noparens{#1}}}}}}}}
\def\sm@parens#1{\@ifnextchar\bgroup%
  {\mathchoice{\@dsm{#1}}{\@tsm{#1}}{\@tsm{#1}}{\@tsm{#1}}\sm@parens}%
  {\@ifnextchar\prd%
    {\mathchoice{\@dsm{#1}}{\@tsm{#1}}{\@tsm{#1}}{\@tsm{#1}}\@eatprd}%
    {\mathchoice{\@dsm{#1}}{\@tsm{#1}}{\@tsm{#1}}{\@tsm{#1}}}}}
\def\@eatprd\prd{\prd@parens}
\def\sm@noparens#1{\mathchoice{\@dsm@noparens{#1}}{\@tsm{#1}}{\@tsm{#1}}{\@tsm{#1}}}
\def\lsm#1{\@ifnextchar\bgroup{\@lsm{#1}\lsm}{\@@lsm{#1}}}
\def\@lsm#1{\mathchoice{{\textstyle\sum}}{\sum}{\sum}{\sum}({\textstyle #1})\;}
\def\@@lsm#1{\mathchoice{{\textstyle\sum}}{\sum}{\sum}{\sum}({\textstyle #1}),\ }
\def\tsm#1{\@tsm{#1}\@ifnextchar\bgroup{\tsm}{}}
\def\@tsm#1{\mathchoice{{\textstyle\sum_{(#1)}}}{\sum_{(#1)}}{\sum_{(#1)}}{\sum_{(#1)}}}
\def\dsm#1{\@dsm{#1}\@ifnextchar\bgroup{\dsm}{}}
\def\@dsm#1{\sum_{(#1)}\,}
\def\@dsm@noparens#1{\sum_{#1}\,}


%%%% THEOREM ENVIRONMENTS %%%%

% The cleveref package provides \cref{...} which is like \ref{...}
% except that it automatically inserts the type of the thing you're
% referring to, e.g. it produces "Theorem 3.8" instead of just "3.8"
% (and hyperref makes the whole thing a hyperlink).  This saves a slight amount
% of typing, but more importantly it means that if you decide later on
% that 3.8 should be a Lemma or a Definition instead of a Theorem, you
% don't have to change the name in all the places you referred to it.

% The following hack improves on this by using the same counter for
% all theorem-type environments, so that after Theorem 1.1 comes
% Corollary 1.2 rather than Corollary 1.1.  This makes it much easier
% for the reader to find a particular theorem when flipping through
% the document.
\makeatletter
\def\defthm#1#2#3{%
  %% Ensure all theorem types are numbered with the same counter
  \newaliascnt{#1}{thm}
  \newtheorem{#1}[#1]{#2}
  \aliascntresetthe{#1}
  %% This command tells cleveref's \cref what to call things
  \crefname{#1}{#2}{#3}% following brace must be on separate line to support poorman cleveref sed file
}

% Now define a bunch of theorem-type environments.
\newtheorem{thm}{Theorem}[section]
\crefname{thm}{Theorem}{Theorems}
%\defthm{prop}{Proposition}   % Probably we shouldn't use "Proposition" in this way
\defthm{cor}{Corollary}{Corollaries}
\defthm{lem}{Lemma}{Lemmas}
\defthm{axiom}{Axiom}{Axioms}
% Since definitions and theorems in type theory are synonymous, should
% we actually use the same theoremstyle for them?
\theoremstyle{definition}
\defthm{defn}{Definition}{Definitions}
\theoremstyle{remark}
\defthm{rmk}{Remark}{Remarks}
\defthm{eg}{Example}{Examples}
\defthm{egs}{Examples}{Examples}
\defthm{notes}{Notes}{Notes}

% Display format for sections
\crefformat{section}{\S#2#1#3}
\Crefformat{section}{Section~#2#1#3}
\crefrangeformat{section}{\S\S#3#1#4--#5#2#6}
\Crefrangeformat{section}{Sections~#3#1#4--#5#2#6}
\crefmultiformat{section}{\S\S#2#1#3}{ and~#2#1#3}{, #2#1#3}{ and~#2#1#3}
\Crefmultiformat{section}{Sections~#2#1#3}{ and~#2#1#3}{, #2#1#3}{ and~#2#1#3}
\crefrangemultiformat{section}{\S\S#3#1#4--#5#2#6}{ and~#3#1#4--#5#2#6}{, #3#1#4--#5#2#6}{ and~#3#1#4--#5#2#6}
\Crefrangemultiformat{section}{Sections~#3#1#4--#5#2#6}{ and~#3#1#4--#5#2#6}{, #3#1#4--#5#2#6}{ and~#3#1#4--#5#2#6}

% Also number formulas with the theorem counter
\let\c@equation\c@thm
\numberwithin{equation}{section}

% Local Variables:
% mode: latex
% TeX-master: "higher-inductive-semantics"
% End:

\autodefs{\fPush\sFib}
\def\name#1{\ulcorner #1\urcorner}
\let\J\iJ
\let\W\iW
\let\F\cF
\let\N\iN
\mdef\Mf{\sM_{\mathbf{f}}}
\mdef\fibm{(\sM,\cF)}
\mdef\fibmf{(\Mf,\cF_{\mathbf{f}})}
\mdef\fibmbang{(\sM,\cF_!)}
\mdef\fibmfbang{(\Mf,\cF_{\mathbf{f},!})}
\let\C\cC
\let\T\cT
\let\r\ir
\def\zero{\mathsf{zero}}
\def\succ{\mathsf{succ}}
\def\nrec{\mathsf{nrec}}
\def\fold{\mathsf{fold}}
\def\wrec{\mathsf{wrec}}
\autodefs{\fId}
\let\Id\fId
\let\G\cG
\def\dG{\widehat{\G}}
\let\type\fibtype
\def\drefl{\refl'}
\usepackage{scalefnt}
\newcommand{\mysc}[1]{\text{\scalefont{0.76}\uppercase{#1}}}
\renewcommand{\idover}[4][]{\fId_{#1}^{#4}(#2,#3)}
\newcommand{\Idtwo}[2]{\widetilde{\Id^{#1}_{#2}}}
\newcommand{\rtwo}[2]{\widetilde{\r^{#1}_{#2}}}
\newcommand{\idovertwo}[4][]{\widetilde{\fId^{#4}_{#1}}(#2,#3)}
\newcommand{\aptwo}[1]{\widetilde{\ap_{#1}}}
\setcounter{tocdepth}{1}
\usepackage[utf8]{inputenc}
\hyphenation{pseu-do-func-tor-ial}
\tikzset{idmap/.style={double equal sign distance,-}}

\title{Semantics of pushouts in homotopy type theory}

\author{Peter LeFanu Lumsdaine}
\author{Michael Shulman}

\begin{document}

\maketitle
\tableofcontents

\section{Introduction}
\label{sec:introduction}



\section{Good model categories}
\label{sec:model-categ-type}

We will work throughout in the following context.

\begin{defn}
  A \textbf{good model category} is a model category \sM with the following additional properties.
  \begin{enumerate}
  \item \sM is simplicial.\label{item:m0}
  \item The cofibrations in \sM are exactly the monomorphisms.\label{item:m1}
  \item \sM is right proper, i.e.\ weak equivalences are preserved by pullback along fibrations.\label{item:m2}
  \item As a category, \sM is locally cartesian closed.\label{item:m3}
  \end{enumerate}
\end{defn}

In particular, a \emph{Cisinski model category}~\cite{cisinski:topos,cisinski:presheaves} is a model structure on a Grothendieck topos whose cofibrations are the monomorphisms.
Therefore, any right proper simplicial Cisinski model category is a good model category.

Assumption~\ref{item:m0} tells us that \sM is simplicially enriched, with powers and copowers\footnote{Also known as cotensors and tensors, respectively.} satisfying the pushout-product and pullback-corner axioms.
This enables us to construct path objects (which model identity types) as simplicial powers.
Specifically, if $\ivl$ denotes the 1-simplex (the simplicial interval), then for any fibrant object $A$, the power $A^\ivl$ is a path-object for $A$.
Likewise, if $p:A\fib \Gamma$ is a fibration, then the ``local power'' $B^\ivl_A = A^\ivl \times_{\Gamma^\ivl} \Gamma$ is a path-object for $A$ regarded as a type over $\Gamma$; see \cref{thm:stable-id}.

This is important for us because it means that homotopies can be represented in adjoint form: a map $A\to B^\ivl$ is equivalent to a map $A\ten \ivl \to B$, where $A\ten \ivl$ is the simplicial copower.
Note that like other colimits, the simplicial copower of $p:A\fib \Gamma$ in the slice category over $\Gamma$ is just $A\ten\ivl$ with the projection $A\ten\ivl \to A \to \Gamma$.
Moreover, each slice category of \sM is a simplicial model category, and pullback preserves both simplicial powers and copowers.

Every good model category is a type-theoretic model category in the sense of~\cite{shulman:invdia}, and hence the subcategory of fibrant objects is a type-theoretic fibration category.
In particular, cofibrations are stable under pullback, so acyclic cofibrations are stable under pullback along fibrations, and hence dependent products of fibrations along fibrations are fibrations.
This gives the categorical analogue of the structure required to model type theory with dependent sums (including a unit type), dependent products, and identity types.

To actually construct such an interpretation requires a coherence theorem\footnote{And also an ``initiality theorem''.} making all the structure strictly stable under pullback.
We will use the coherence method of~\cite{lw:localuniv}, which applies to many different kinds of structure.
The input to this method is a \emph{comprehension category} $(\C,\T)$.
One then defines the \textbf{left adjoint splitting} $(\C,\T_!)$ (or just $\C_!$ for short) in which a type $A\in \T_!(\Gamma)$ consists of an object $V_A\in\C$, a type $E_A\in\T(V_A)$, and a map $\name{A}:\Gamma\to V_A$.
We call $V_A$ the ``local universe'' and think of this as a representative of the pullback of $E_A$ along $\name{A}$.
Reindexing of such types is done by simple composition ($A[f]$ is $\name{A}\circ f$) which is strictly associative.
The local universes technique then shows that if $\C$ satisfies a technical condition (see~\eqref{eq:lf} below) and $\C$ has ``weakly stable'' structure of some sort, meaning that it exists in each fiber $\T(\Gamma)$ and the reindexing of \emph{a} structure is \emph{a} structure, then $\C_!$ admits \emph{strictly} stable structure (obtained by constructing weakly stable structure once in the ``universal case''), and thus (modulo an appropriate ``initiality theorem'') provides semantics for the corresponding type-theoretic rules.
See~\cite{lw:localuniv} for details.

We summarize the above discussion as follows.

\begin{thm}
  Any good model category $\sM$ admits a natural structure of a comprehension category $\fibm$ where $\cF(\Gamma)$ is the category of fibrations with codomain $\Gamma$.
  Moreover, $\fibm$ has weakly stable dependent sums, unit type, dependent products, and identity types, and therefore $\fibmbang$ has strictly stable dependent sums, unit type, dependent products, and identity types.
\end{thm}


% We will also need the following enhancement of~\cite{lw:localuniv}.
% For a type-theoretic model category, by the \textbf{fibrant left adjoint splitting} we mean the substructure of the left adjoint splitting consisting of only those types whose local universe object $V_A$ is fibrant.
% Up to homotopy, this carries the same information, since every map is weakly equivalent to a fibration between fibrant objects.

% \begin{thm}\label{thm:fibrant-lu}
%   In a type-theoretic fibration category, the fibrant left adjoint splitting also inherits strictly stable structure of dependent sums, dependent products, and identity types.
% \end{thm}
% \begin{proof}
%   It suffices to show that the ``universal case'' constructions on local universes all preserve fibrancy.
%   The product of fibrant objects is certainly fibrant, and the construction $V_A \triangleleft V_B$ from~\cite[\S3.3]{lw:localuniv} is fibrant if all inputs are since the dependent product of a fibration along a fibration is a fibration.
%   This suffices for the formation rules of dependent sums and products, and the other rules are automatically stable since they are unique (in our context these objects have strict universal properties).

%   For the identity type, the local universe of the formation and introduction rules is just the domain of a composite of fibrations over a fibrant object, hence fibrant.
%   Here we need to consider also the local universe for the elimination and computation rules; but this is also constructed as a dependent product of a fibration along a fibration.
% \end{proof}



\section{Coproducts}
\label{sec:coproducts}

We warm up for pushouts by considering the case of coproducts.
To start with, we specialize the definitions from~\cite[\S3.4.1]{lw:localuniv} to our context of a good model category.
We qualify these definitions with ``typal'' (the adjective of ``type'') to distinguish them from the ordinary categorical constructions in \sM.

\begin{defn}\label{defn:sum}
  A \textbf{typal coproduct} of fibrations $A_1\to \Gamma$ and $A_2\to\Gamma$ consists of a fibration $A_1\oplus A_2 \to\Gamma$ with maps $\nu_i: A_i \to A_1\oplus A_2$ over $\Gamma$ such that for any fibration $C\to A_1\oplus A_2$ with sections $t_i : A_i \to C$ over $\nu_i$, there exists a section $s:A_1\oplus A_2 \to C$ such that $s\circ \nu_i = t_i$.

  We say \sM has \textbf{weakly stable typal coproducts} if for any $A_1\to \Gamma$ and $A_2\to\Gamma$ there exists a typal coproduct $A_1\oplus A_2$ such that for any $\sigma:\Delta\to\Gamma$, the pullback $\sigma^*(A_1\oplus A_2)$ with injections $\sigma^*\nu_i$ is a typal coproduct of $\sigma^*A_1$ and $\sigma^* A_2$.
\end{defn}

\begin{thm}[{\cite[Lemma 3.4.1.4]{lw:localuniv}}]
  If \sM has weakly stable typal coproducts, then its left adjoint splitting models type theory with coproduct types.
  % Moreover, so does its fibrant left adjoint splitting.
\end{thm}
% \begin{proof}
%   The first statement follows from~\cite[Lemma 3.4.1.4]{lw:localuniv}.
%   For the second, it remains to show that fibrancy of local universes is preserved.
%   The local universe for the formation and introduction rules is $V_{A_1}\times V_{A_2}$, which is certainly fibrant if $V_{A_1}$ and $V_{A_2}$ are.
%   The local universe for the elimination and computation rules is more complicated, but involves only fibrations and dependent products, so it is also fibrant if all inputs are.
% \end{proof}

Now, how do we actually \emph{construct} weakly stable typal coproducts?
Of course, since \sM is a model category, it has ordinary categorical coproducts, and the coproduct $A_1 + A_2 \to \Gamma$ satisfies all parts of \cref{defn:sum} except that it may not be a fibration.
(Admittedly, for some particularly nice \sM, such as simplicial sets, it is always a fibration.
However, we treat the general case not only out of a desire for generality, but because in the case of pushouts the analogous argument will be necessary even when \sM is simplicial sets.)

The obvious solution is to fibrantly replace it.
Thus, let $A_1 + A_2 \to A_1\oplus A_2 \fib \Gamma$ be an (acyclic cofibration, fibration) factorization.

\begin{thm}\label{thm:coproduct}
  For any fibrations $A_1 \fib \Gamma$ and $A_2\fib\Gamma$, % with $\Gamma$ a fibrant object, 
  the fibration $A_1\oplus A_2 \fib\Gamma$ is a weakly stable typal coproduct.
\end{thm}
\begin{proof}
  We define the injections by composition with those of $A_1+A_2$.
  To show that $A_1\oplus A_2$ is a typal coproduct, let $C\fib A_1\oplus A_2$ be a fibration with sections $t_i$ over $A_i$.
  Then the universal property of $A_1+A_2$ induces a section $t : A_1+A_2 \to C$, and the lifting property of the acyclic cofibration $A_1+A_2 \to A_1\oplus A_2$ against the fibration $C\fib A_1\oplus A_2$ allows us to extend this section to $A_1\oplus A_2$.

  To show that this typal coproduct is weakly stable, it suffices to show that for any $\sigma :\Delta\to\Gamma$, the pullback $\sigma^*(A_1+A_2) \to \sigma^*(A_1\oplus A_2)$ is again an acyclic cofibration.
  Since cofibrations are stable under pullback, this map is a cofibration; thus it remains to show it is a weak equivalence.
  Weak equivalences are not generally stable under pullback, but weak equivalences between fibrations are; so we would be done if $A_1+A_2 \to \Gamma$ were a fibration, but the whole point is that it may not be.

  Right properness of \sM ensures that weak equivalences are also stable under pullback along fibrations, so we would be done if $\sigma$ were a fibration.
  In general it may not be either, but we can factor it as an acyclic cofibration followed by a fibration.
  Thus, without loss of generality we may assume that $\sigma$ is an acyclic cofibration.

  In this case, right properness tells us that the induced maps $\sigma^*A_i \to A_i$ and $\sigma^*(A_1\oplus A_2) \to (A_1\oplus A_2)$ are weak equivalences, hence acyclic cofibrations, since they are pullbacks of $\sigma$ along fibrations.
  Moreover, acyclic cofibrations are closed under coproducts (being the left class of a weak factorization system), so the induced map $\sigma^*A_1 +\sigma^*A_2 \to A_1+A_2$ is an acyclic cofibration.
  Moreover, since $\sigma^*$ is a left adjoint, it preserves coproducts.
  Thus, in the following square
  \[ \xymatrix{ \sigma^*(A_1+A_2) \ar[r]^-\sim \ar[d] & A_1+A_2 \ar[d]^\sim \\ \sigma^*(A_1\oplus A_2) \ar[r]_-\sim & A_1\oplus A_2} \]
  all the marked maps are acyclic cofibrations, hence weak equivalences.
  Hence, by 2-out-of-3, so is the remaining map, which is what we wanted.
\end{proof}

\begin{cor}
  The left adjoint splitting of \sM models coproduct types.\qed
\end{cor}


\section{Pushouts in model categories}
\label{sec:pushouts}

Higher inductive pushouts are not a traditionally standard type constructor, so we need to begin with definitions.
First we consider a sort of weakly stable structure that only makes sense in a context where we have simplicial homotopies.
In \cref{sec:coherence-pushouts} we will rephrase this in type-theoretic language and prove the relevant local universes coherence theorem.

\begin{defn}\label{defn:ivl-pushout}
  Let $f_1:A\to B_1$ and $f_2:A\to B_2$ be morphisms between fibrations $A\fib\Gamma$ and $B_i\fib\Gamma$.
  A \textbf{$\ivl$-typal pushout} is a fibration $D\fib\Gamma$ with maps $\nu_i : B_i \to D$ over $\Gamma$ and a homotopy $\mu:A \ten \ivl \to D$ over $\Gamma$ between $\nu_1 \circ f_1$ and $\nu_2\circ f_2$, such that for any fibration $C\fib D$ equipped with sections $t_i : B_i \to C$ over $\nu_i$ and a homotopy $u:A\ten \ivl \to C$ over $\mu$, there exists a section $s:D\to C$ such that $s\circ \nu_i = t_i$ and $s\circ \mu = u$.

  We say \sM has \textbf{weakly stable $\ivl$-typal pushouts} if for any $f_1,f_2$ there exists a $\ivl$-typal pushout $D$ such that for any $\sigma:\Delta\to\Gamma$, the pullback $\sigma^* D$ with injections $\sigma^*\nu_i$ and homotopy $\sigma^*\mu$ is a $\ivl$-typal pushout of $\sigma^*f_1$ and $\sigma^* f_2$.
\end{defn}

\begin{thm}
  Any good model category has weakly stable $\ivl$-typal pushouts.
\end{thm}
\begin{proof}
  Give $f_1,f_2$, let $Q$ be their explicit homotopy pushout, meaning the pushout of the following diagram in \sM:
  % \[ \xymatrix@-.5pc{ & A \ar[r]^{f_1} \ar[d]^{i_1} & B_1 \\ A \ar[d]_{f_2} \ar[r]^-{i_2} & A\ten \ivl\\ B_2 } \]
  \[ \xymatrix{ A+A \ar[r]^{\iota} \ar[d]_{f_1+f_2} & A\ten\ivl \\ B_1+B_2 } \]
  By construction, this satisfies all parts of \cref{defn:ivl-pushout} except that it may not be a fibration over $\Gamma$.
  (And in this case it really isn't, even in simplicial sets.)
  Let $D$ be its fibrant replacement, i.e.\ we have a factorization $Q\to D \fib \Gamma$ as an acyclic cofibration followed by a fibration.
  Then $D$ has injections and a homotopy obtained by composition from $Q$, and for any fibration $C\fib D$ as in \cref{defn:ivl-pushout} we can first define a section over $Q$ by its universal property and then extend to $D$ by lifting against the acyclic cofibration $Q\to D$.

  It remains to deal with weak stability.
  As in \cref{thm:coproduct}, for this it suffices to show that $\sigma^*Q\to\sigma^*D$ is an acyclic cofibration for any acyclic cofibration $\sigma:\Delta\to\Gamma$, and by 2-out-of-3 it suffices to show that $\sigma^*Q \to Q$ is an acyclic cofibration.
  Again, $\sigma^*$ preserves colimits and simplicial copowers, so $\sigma^*Q $ is the pushout of $\sigma^*B_1+\sigma^*B_2$ and $\sigma^*A \ten\ivl$ under $\sigma^*A+\sigma^*A$.
  Furthermore, again as in \cref{thm:coproduct}, $\sigma^*A \to A$ and $\sigma^*B_i \to B_i$ are acyclic cofibrations, hence so are $\sigma^*A+\sigma^*A \to A+A$ and ${\sigma^*B_1 +\sigma^*B_2} \to B_1+B_2$.

  Now consider the following commutative cube, in which the left-hand and right-hand faces are pushouts, and the objects $R$ and $S$ are also pushouts.
  \[\begin{tikzcd}
    & \sigma^* A+\sigma^*A \arrow[dl] \arrow[rr] \arrow[dd] \ar[dr,phantom,"R"{name=sr}]
    & & A+A \arrow[dl] \arrow[dd] \ar[to=sr] \\
    \sigma^*A\ten\ivl\arrow[dd] \ar[to=sr] & & A\ten\ivl \ar[from=sr] \\
    & \sigma^*B_1+\sigma^*B_2 \arrow[dl] \arrow[rr] \ar[dr,phantom,"S"{name=tr}]
    \ar[from=sr,to=tr,crossing over]
    \arrow[from=ul,to=ur, crossing over]
    & & B_1+B_2 \arrow[dl] \ar[to=tr] \\
    \sigma^*Q \arrow[rr] \ar[to=tr] & & Q \arrow[from=uu, crossing over]
    \ar[from=tr] \\
  \end{tikzcd}\]
  To show that $\sigma^*Q \to Q$ is an acyclic cofibration, it will suffice to show that both of its factors $\sigma^*Q\to S$ and $S\to Q$ are such.
  The former is easy, since it is a pushout of the acyclic cofibration ${\sigma^*B_1 +\sigma^*B_2} \to B_1+B_2$.
  For the latter, a standard argument shows that it is the pushout of the map $R\to A\ten\ivl$ in the upper square.
  However, this map is the pushout-product of the acyclic cofibration $\sigma^*A\to A$ and the cofibration of simplicial sets $\mathbf{2}\to\ivl$.
  Thus since the model structure is simplicial, it is an acyclic cofibration as well.
\end{proof}


\section{Pushouts in comprehension categories}
\label{sec:coherence-pushouts}

Inside of (ordinary, Martin-L\"of) type theory, of course, we do not have a ``$\ivl$'', so \cref{defn:ivl-pushout} does not correspond directly to anything type-theoretic the way \cref{defn:sum} does.
Instead we need a version of this definition that refers only to identity types, which categorically means path-objects.
This is an instance of ``one type constructor stacked on top of another'', like the case of $\mathsf{W}$-types considered in~\cite[\S3.4.4]{lw:localuniv}; hence we need to start by defining good classes of identity types.

Let $(\C,\T)$ be a comprehension category.
As in~\cite{lw:localuniv}, if $A\in\T(\Gamma)$ we denote its comprehension by $\Gamma.A\to\Gamma$, and its reindexing along $\sigma:\Delta\to\Gamma$ by $A[\sigma]$.

In what follows, by a \emph{family} we mean an \emph{indexed family}.
That is, a ``family of elements of $S$'', for any set $S$, consists of a set $F$ and a function $F\to S$.
We often abuse notation by identifying elements of $F$ with their images in $S$, but it is important that such a family is not just a subset of $S$.
There is a category $\mathrm{Fam}$ whose objects are families $F\to S$ and whose morphisms are commutative squares.

\begin{defn}\label{defn:id}
  A \textbf{stable class of identity types} on \C consists of:
  \begin{itemize}
  \item For each $A\in \T(\Gamma)$, a non-empty family $\G_\Id(A)$ of elements of $\T(\Gamma.A.A)$, called ``good identity types'' $\Id_A$.
    These must be weakly stable under reindexing, in that for any $\sigma:\Delta\to\Gamma$, there is a morphism of families $\G_\Id(A) \to \G_\Id(A[\sigma])$ over the reindexing functor $\T(\Gamma.A.A) \to \T(\Delta.A[\sigma].A[\sigma])$.
    If $\Id_A\in\G_\Id(A)$ is a good identity type for $A$, we abuse notation by writing $\Id_A[\sigma] \in \G_\Id(A[\sigma])$ for its image.
  \item For each good identity type $\Id_A\in\G_\Id(A)$, a non-empty family $\G_\r(A,\Id_A)$ of ``good reflexivity terms'' that are sections of $\Id_A[\delta_A] \in \T(\Gamma.A)$.
    These must be weakly stable under reindexing, in that we have morphisms of families $\G_\r(A,\Id_A) \to \G_\r(A[\sigma],\Id_A[\sigma])$ over the reindexing morphisms of types and good identity types.
  \item For each good identity type $\Id_A$ and good reflexivity term $\r$, and each type $C\in\T(\Gamma.A.A.\Id_A)$ equipped with a section $c$ of $C[\delta_A,\r]\in\T(\Gamma.A)$, a non-empty family $\G_\J(\Id_A,\r,C,c)$ of ``good extensions'' of $c$ to a section of $C$ itself.
    Of course, every good extension must actually be an extension, i.e.\ its composite with $\delta_A \circ \r$ must be $c$.
    Good extensions must also be weakly stable under reindexing, in that we have morphisms $\G_\J(\Id_A,\r,C,c) \to \G_\J(\Id_A[\sigma],\r[\sigma],C[\sigma],c[\sigma])$ over the reindexings of everything else.
  \end{itemize}
\end{defn}

Note that the underlying data of a good identity type and a good reflexivity term consist equivalently of a factorization of $\Gamma.A \to \Gamma.A.A$ through the comprehension $\Gamma.A.A.\Id_A \to \Gamma.A.A$ of some type $\Id_A$.

\begin{thm}\label{thm:stable-id}
  If \sM is a good model category, then \fibm has a stable class of identity types, called the \textbf{canonical stable class of identity types}, in which:
  \begin{itemize}
  \item $\G_\Id(A)$ is the set of objects of $\sM/\Gamma$ equipped with data exhibiting the universal property of the simplicial power $(\Gamma.A)^\ivl_\Gamma$ therein.
  \item For each such object, the factorization $\Gamma.A \to (\Gamma.A)^\ivl_\Gamma \to \Gamma.A.A \cong \Gamma.A \times_\Gamma \Gamma.A$ is induced by powering with the maps $\mathbf{2} \to \ivl \to \mathbf{1}$ of simplicial sets.
    In particular, every good identity type has exactly one good reflexivity term.
  \item For any $C$ and $c$, every extension of $c$ to $\Id_A$ is good in a unique way.
  \end{itemize}
\end{thm}
\begin{proof}
  The projection $(\Gamma.A)^\ivl_\Gamma \to \Gamma.A\times_\Gamma \Gamma.A$ is the pullback corner map for the fibrant object $\Gamma.A\in\sM/\Gamma$ and the cofibration $\mathbf{2}\to \ivl$.
  Since $\sM/\Gamma$ is a simplicial model category, this map is a fibration, hence the comprehension of a type over $\Gamma.A.A$.

  The stability of simplicial powers under pullback gives the reindexing operations for good identity types and good reflexivity terms.
  Note that unlike for many other type constructors such as $\Sigma$- and $\Pi$-types, although these identity types are determined by a 1-categorical universal property, the type-theoretic data we consider (the reflexivity term) is not sufficient to describe this universal property.

  Similarly, the inclusion $\Gamma.A\to (\Gamma.A)^\ivl_\Gamma$ is the pullback corner map for $A$ and the projection $\ivl \to \mathbf{1}$.
  Since $\Gamma.A$ is fibrant in $\sM/\Gamma$ and $\ivl \to \mathbf{1}$ is a weak equivalence between fibrant objects, this is a weak equivalence.
  Moreover, it is a split monomorphism, hence a cofibration, and thus an acyclic cofibration.
  It follows that given any $C$ and $c$ there exists such an extension, i.e.\ the family of good extensions is non-empty.
  Of course the reindexing of any extension is again an extension, and pseudofunctoriality is automatic.
\end{proof}

By the adjointness between simplicial powers and copowers, a simplicial homotopy $A\ten\ivl \to B$ between $f,g:A\toto B$ is equivalently a lift of $(f,g):A\to B\times B$ to any canonical identity type $\Id_B = B^\ivl$, i.e.\ a term of $\Id_B[(f,g)]$ in context $A$.
But in order to rephrase \cref{defn:ivl-pushout} relative to a stable class of identity types, we also need to talk about ``homotopies over homotopies'', for which we need \emph{dependent identity types}.
Inside of type theory, the dependent identity type looks like this, for a dependent type $x:A \types B(x)\type$:
\[ a_1:A, a_2:A, e:\id[A]{a_1}{a_2}, b_1:A(a_1), b_2:A(a_2) \types \idover[x.B(x)]{b_1}{b_2}{e} \type \]
That is, it tells us how to identify two points in different fibers along a path in the base.
Inside type theory, there are many ways to define such a type:
\begin{enumerate}\label{idover}
\item If we first define the \emph{transport} operation $\transf{e}:B(x) \to B(y)$ (using the eliminator for identity types), then we can define $\idover[x.B(x)]{u}{v}{e}$ to be $\id[B(y)]{\transf{e}(u)}{v}$.
  This is the definition used by~\cite{hottbook} and~\cite{hottcoq,bglsss:hottcoq}.\label{item:idover1}
\item We could instead use $\id[B(x)]{u}{\transf{(\opp{e})}(v)}$, where $\opp{e}:\id[A]{y}{x}$ is the inverse path of $e$ (i.e.\ $\opp{(\blank)}$ witnesses the symmetry of equality).\label{item:idover2}
\item We could use the eliminator for identity types on $e$, with $\idover[x.B(x)]{u}{v}{\refl_x}$ defined to be $\id[B(x)]{u}{v}$.
  This is the definition used by~\cite{hottagda}.\label{item:idover3}
\item We could define $\idover[x.B(x)]{u}{v}{e}$ as an inductive family, with a single constructor giving for any $x:A$ and $u:B(x)$ an element $\refl_u : \idover[x.B(x)]{u}{u}{\refl_x}$.\label{item:idover4}
\end{enumerate}
Compared to the first three, option~\ref{item:idover4} has the disadvantage that $\idover[x.B(x)]{u}{v}{\refl_x}$ is not judgmentally equal to $\id[B(x)]{u}{v}$.
However, option~\ref{item:idover4} is also the one that corresponds most directly to the native path-object structure in a good model category, so it is the one we will adopt.
In \cref{sec:pushouts-type-theory} we will show that our construction also gives pushouts relative to choices~\ref{item:idover1}--\ref{item:idover3}, albeit in a slightly weaker sense.

\begin{defn}\label{defn:dep-id}
  Suppose \C has a stable class of identity types.
  A \textbf{stable class of dependent identity types} relative to this stable class consists of the following data, each of which must be weakly stable in a straightforward sense.
  \begin{itemize}
  \item For each $A\in\T(\Gamma)$ and $B\in\T(\Gamma.A)$, and each good identity type $\Id_A\in \G_\Id(A)$, a non-empty family $\dG_\Id(A,\Id_A,B)$ of ``good dependent identity types'' $\Id^A_B$ in $\T(\Gamma.A.A.\Id_A.B.B)$.
  \item For each good dependent identity type as above, and each good reflexivity term $\r_A$ for $\Id_A$, a non-empty family $\dG_\r(A,\Id_A,\r,B,\Id^A_B)$ of ``good dependent reflexivity terms'' that are sections of $\Id^A_B[\delta_A,\r_A,\delta_B] \in \T(\Gamma.A.B)$.
  \item For each good dependent identity type and good dependent reflexivity term as above, and each type $C\in \T(\Gamma.A.A.\Id_A.B.B.\Id^A_B)$ equipped with a section $c$ of $C[\delta_A,\r_A,\delta_B,\r^A_B]\in \T(\Gamma.A.B)$, a nonempty family $\dG_\J(A,\Id_A,\r_A,B,\Id^A_B,\r^A_B)$ of ``good extensions'' of $c$ to a section of $C$ itself (which are actually extensions thereof).
  \end{itemize}
\end{defn}

Similarly to the non-dependent case, the underlying data of a good dependent identity type and a good dependent reflexivity term consist of a factorization of the diagonal of $\Gamma.A.B$ that lies over a given factorization of the diagonal of $\Gamma.A$, such that the dashed pullback map shown below is a dependent projection:
\[
\begin{tikzcd}[row sep=huge,column sep=huge]
  \Gamma.A.B \ar[r] \ar[d] & \Gamma.A.A.\Id_A.B.B.\Id^A_B \ar[rr] \ar[d] \ar[drr,phantom,near start,"\Gamma.A.A.\Id_A.B.B"{name=hi}] \ar[dashed,to=hi] && \Gamma.A.A.B.B \ar[d] \ar[from=hi] \\
  \Gamma.A \ar[r] & \Gamma.A.A.\Id_A \ar[rr] \ar[from=hi] && \Gamma.A.A
\end{tikzcd}
\]

\begin{thm}\label{thm:stable-dep-id}
  If \sM is a good model category, then \fibm has a \textbf{canonical stable class of dependent identity types} over the canonical stable class of identity types, in which:
  \begin{itemize}
  \item $\dG_\Id(A,\Id_A,B)$ is the set of objects of $\sM/\Gamma$ equipped with the universal property of a simplicial power $(\Gamma.A.B)^\ivl_\Gamma$.
  \item For each such object, the above factorization is given by the following diagram:
    \[
    \begin{tikzcd}[row sep=large]
      \Gamma.A.B \ar[r] \ar[d] & (\Gamma.A.B)^\ivl_\Gamma \ar[rr] \ar[d] \ar[drr,phantom,near start,"\bullet"{name=hi}] \ar[dashed,to=hi] && \Gamma.A.A.B.B \ar[d] \ar[from=hi] \\
      \Gamma.A \ar[r] & (\Gamma.A)^\ivl_\Gamma \ar[rr] \ar[from=hi] && \Gamma.A.A
    \end{tikzcd}
    \]
    In particular, every good dependent identity type has a unique good dependent reflexivity term.
  \item for any $C$ and $c$, every extension of $c$ to $\Id^A_B$ is good.
  \end{itemize}
\end{thm}
\begin{proof}
  Similarly to the non-dependent case, the dotted projection above is the pullback corner map for the fibration $\Gamma.A.B\to\Gamma.A$ and the cofibration $\mathbf{2}\to \ivl$ in the simplicial model category $\sM/\Gamma$, hence a fibration.
  The inclusion $\Gamma.A.B \to (\Gamma.A.B)^\ivl_\Gamma$ is an acyclic cofibration for the same reasons as in \cref{thm:stable-id}, so the set of good extensions is non-empty.
\end{proof}

Finally, in \cref{defn:ivl-pushout} we also need to postcompose a homotopy with a dependent function.
That is, given $x:A \types f(x):B(x)$, we need the following term:
\[ a_1:A, a_2:A, e:\id[A]{a_1}{a_2} \types \ap_f(e) : \idover[x.B(x)]{e}{f(a_1)}{f(a_2)} \]
Inside type theory there is a standard way to define $\ap$, defined using the eliminator for the identity type.
However, once again, in a good model category there is a different way to define it using the functoriality of simplicial powers (see \cref{thm:stable-ap}).
Thus, we also introduce this abstractly.

\begin{defn}\label{defn:ap}
  Suppose \C has stable classes of identity types and dependent identity types.
  A \textbf{stable class of identity applications} consists of, for each types $A\in\T(\Gamma)$ and $B\in\T(\Gamma.A)$, each section $f$ of $B$, each good identity type and reflexivity term for $A$, and each corresponding dependent identity type and dependent reflexivity term for $B$, a non-empty family of ``good identity applications'', which are sections $\ap_f$ of the projection $\Gamma.A.A.\Id_A.B.B.\Id^A_B \to \Gamma.A.A.\Id_A$ such that the following squares commute:
  \[
  \begin{tikzcd}
    \Gamma.A.B \ar[r,"\r"] & \Gamma.A.A.\Id_A.B.B.\Id^A_B \ar[r] & \Gamma.A.A.B.B \\
    \Gamma.A \ar[r,"\r"] \ar[u,"f"] & \Gamma.A.A.\Id_A \ar[u,"{\ap_f}"] \ar[r] & \Gamma.A.A \ar[u,"{(f,f)}",swap]
  \end{tikzcd}
  \]
  Moreover, good identity applications must be weakly stable under reindexing in an evident sense.
\end{defn}

\begin{thm}\label{thm:stable-ap}
  If \sM is a good model category, then \fibm has a \textbf{canonical stable class of identity applications} over the canonical stable classes of identity types and dependent identity types, in which the good sections $\ap_f$ are the maps $f^\ivl_\Gamma$ induced by the functoriality of simplicial powers.
\end{thm}
\begin{proof}
  The requisite squares commute by the two-variable functoriality of simplicial powers.
\end{proof}

Note that in order for these maps to be well-defined, we must require an object of $\G_\Id(A)$ to be \emph{equipped with} the structure of a simplicial power, and hence $\G_\Id(A)$ cannot be merely a sub\emph{set} of $\T(\Gamma.A.A)$, since this structure is not determined even by the reflexivity term.

Finally, we can define a type-theoretic notion of pushout.

\begin{defn}\label{defn:pushout}
  Suppose $\C$ has stable classes of identity types, dependent identity types, and identity applications.
  A \textbf{stable class of typal pushouts} relative to these stable classes consists of the following data, all weakly stable under reindexing.
  \begin{itemize}
  \item For each pair $f_1:\Gamma.A\to \Gamma.B_1$ and $f_2:\Gamma.A\to \Gamma.B_2$ of morphisms over $\Gamma$, a non-empty family of ``good pushout types'' $D\in\T(\Gamma)$ equipped with ``good injections'' $\nu_i : \Gamma.B_i \to \Gamma.D$ over $\Gamma$.
  \item For any good pushout $D$ and good injections $\nu_i$, and any good identity type for $D$, a non-empty family of ``good glueing data'' maps $\mu : \Gamma.A \to \Gamma.D.D.\Id_D$ over $(\nu_1 f_1,\nu_2 f_2)$.
  \item For any good identity type, good reflexivity term, and good gluing data and good injections for a good pushout $D$, and any type $C\in\T(D)$ equipped with sections $t_i$ over $\nu_i$, a good dependent identity type $\Id^D_C$ with good reflexivity term, and a morphism $u:A\to \Gamma.D.D.\Id_D.C.C.\Id^D_C$ over $(\mu,t_1,t_2)$, we have a non-empty family of good sections $s$ of $C$ such that $s \circ \nu_i = t_i$, and for any good identity application $\ap_s$ we have $\ap_s \circ \mu = u$.
  \end{itemize}
\end{defn}

\begin{thm}\label{thm:model-pushout}
  In any good model category, the $\ivl$-typal pushouts (\cref{defn:ivl-pushout}) are a stable class of typal pushouts relative to the canonical stable classes of identity types, dependent identity types, and identity applications.
\end{thm}
\begin{proof}
  The canonical stable classes were constructed exactly so that the data of \cref{defn:pushout} reduces to that of \cref{defn:ivl-pushout}.
\end{proof}


\section{Pushouts in type theory}
\label{sec:pushouts-type-theory}

Finally, we move to strict stability and type theory.
As in~\cite{lw:localuniv}, we do not formally make the connection all the way to syntactic type theory; we stop with strictly stable structure on a split comprehension category.
(The gap requires an ``initiality theorem'' for syntax, which is straightforward but tedious and requires very different techniques.)

Recall the basic condition on a comprehension category $(\C,\T)$ that makes local universe theorems work:
\begin{equation}
  \label{eq:lf}
  \tag{LF} \parbox{10cm}{$\C$ has finite products, and dependent exponentials of display maps and product projections along display maps.}
\end{equation}

Of course, since a good model category $\sM$ is locally cartesian closed, $\fibm$ always satisfies condition~\eqref{eq:lf}.

\begin{defn}
  A split comprehension category $\C$ has:
  \begin{itemize}
  \item \textbf{strictly stable identity types} if it has a stable class of identity types (\cref{defn:id}) for which each family of good structures is a singleton.
  \item \textbf{strictly stable dependent identity types} if in addition it has a stable class of dependent identity types for which each family of good structures is a singleton.
  \item \textbf{strictly stable identity applications} if in addition it has a stable class of identity applications for which each family of good structures is a singleton.
  \item \textbf{strictly stable typal pushouts} if in addition it has a stable class of typal pushouts for which each family of good structures is a singleton.
\end{itemize}
\end{defn}

Our strictly stable identity types are easily seen to be equivalent to~\cite[Definition 3.4.3.1]{lw:localuniv} with the ``Frobenius'' condition omitted (which we don't bother with since our intended models all have $\Pi$-types).
Modulo the initiality theorem, the above strictly stable structure corresponds to the type-theoretic rules shown in \cref{fig:id,fig:depid,fig:ap,fig:pushouts}.

\begin{rmk}\label{rmk:rule-style}
  Because we are not concerned with syntactic properties such as admissibility of substitution, we have no qualms about adding variables to the context of the conclusion.
  As a result, multiple rules often have the same premises; we combine these by listing multiple conclusions in the same rule.
  This has the additional advantage that each multiple-conclusion rule corresponds to one new local universe on the semantic side.
  (On the other hand, unlike in some syntactic presentations, we do include all type judgment premises in all rules, since these are necessary semantically.)
\end{rmk}

\begin{figure}
  \centering
  \begin{mathpar}
    \inferrule{\Gamma\types A\type} % \\ \Gamma\types a_1:A \\ \Gamma\types a_2:A}
    {\Gamma,a_1:A, a_2:A\types \id[A]{a_1}{a_2}\type \\ %}\and
      % \inferrule{\Gamma\types A\type}{
      \Gamma,a:A \types \refl_a : \id[A]{a}{a}}\and
    \inferrule{\Gamma\types A\type \\ \Gamma,x:A,y:A,e:\id[A]{x}{y}\types C\type \\ \Gamma,x:A \types c:C[x/y,\refl_x/e]} % \\\\ \Gamma \types a_1:A \\ \Gamma\types a_2:A \\ \Gamma\types p:\id[A]{a_1}{a_2}}
    {\Gamma,a_1:A,a_2:A,p:\id[A]{a_1}{a_2}\types \J(xye.C,x.c,a_1,a_2,p):C[a_1/x,a_2/y,p/e] \\ %}\and
    % \inferrule{\Gamma\types A\type \\ \Gamma,x:A,y:A,e:\id[A]{x}{y}\types C\type \\ \Gamma,x:A \types c:C[x/y,\refl_x/e]} % \\ \Gamma \types a:A}
    \Gamma,a:A\types \J(xye.C,x.c,a,a,\refl_a) \jdeq c[a/x]}
  \end{mathpar}
  \caption{Identity types}
  \label{fig:id}
\end{figure}

\begin{figure}
  \centering
  \begin{mathpar}
    \inferrule{\Gamma\types A\type \\ \Gamma,x:A \types B\type}
    {\Gamma, a_1:A, a_2:A, e:\id[A]{a_1}{a_2}, b_1:B[a_1/x], b_2:B[a_2/x] \types \idover[B]{b_1}{b_2}{e}\type \\ %}\and
      % \inferrule{\Gamma\types A\type \\ \Gamma,x:A \types B\type}
    \Gamma, a:A, b:B[a/x]\types \drefl_b : \idover[B]{b}{b}{\refl_a}}\and
    \inferrule{\Gamma\types A\type \\ \Gamma,x:A \types B\type \\ \Gamma,x:A,y:A,e:\id[A]{x}{y}, u:B, v:B[y/x], d:\idover[B]{u}{v}{e} \types C\type \\ \Gamma,x:A,u:B \types c:C[x/y,\refl_x/e,u/v,\drefl_u/d]}
    %\\\\ \Gamma \types a_1:A \\ \Gamma\types a_2:A \\ \Gamma\types p:\id[A]{a_1}{a_2} \\\\
    %\Gamma \types b_1:B[a_1/x] \\ \Gamma\types b_2:B[a_2/x] \\ \Gamma\types q:\idover[B]{b_1}{b_2}{p}}
    {\Gamma, a_1:A, a_2:A, p:\id[A]{a_1}{a_2}, b_1:B[a_1/x], b_2:B[a_2/x], q:\idover[B]{b_1}{b_2}{p} \\ \hspace{2cm}\types \J'(xyeuvd.C,xu.c,a_1,a_2,p,b_1,b_2,q):C[a_1/x,a_2/y,p/e,b_1/u,b_2/v,q/d] \\ %\and
    %\inferrule{\Gamma\types A\type \\ \Gamma,x:A \types B\type \\ \Gamma,x:A,y:A,e:\id[A]{x}{y}, u:B, v:B[y/x], d:\idover[B]{u}{v}{e} \types C\type \\ \Gamma,x:A,u:B \types c:C[x/y,\refl_x/e,u/v,\drefl_u/d]} % \\\\ \Gamma \types a:A \\ \Gamma \types b:B[a/x]
    \Gamma, a:A, b:B[a/x]\types \J'(xyeuvd.C,xu.c,a,a,\refl_a,b,b,\drefl_b) \jdeq c[a/x,b/u]}
  \end{mathpar}
  \caption{Dependent identity types}
  \label{fig:depid}
\end{figure}

\begin{figure}
  \centering
  \begin{mathpar}
    \inferrule{\Gamma\types A\type \\ \Gamma,x:A\types B\type \\ \Gamma,x:A \types f:B} % \\\\ \Gamma\types a_1:A \\ \Gamma\types a_2:A \\ \Gamma\types p:\id[A]{a_1}{a_2}}
    {\Gamma,a_1:A, a_2:A, p:\id[A]{a_1}{a_2} \types \ap_{x.f}(a_1,a_2,p) : \idover[B]{f[a_1/x]}{f[a_2/x]}{p} \\ %}\and
    %\inferrule{\Gamma\types A\type \\ \Gamma,x:A\types B\type \\ \Gamma,x:A \types f:B}{
      \Gamma,a:A\types \ap_{x.f}(a,a,\refl_a)\jdeq \drefl_{f[a/x]}}\and
  \end{mathpar}
  \caption{Identity applications}
  \label{fig:ap}
\end{figure}

\begin{figure}
  \centering
  \begin{mathpar}
    \inferrule{\Gamma\types A\type \\ \Gamma\types B_1\type \\ \Gamma\types B_2\type \\\\
      \Gamma,x:A\types f_1:B_1 \\ \Gamma,x:A \types f_2:B_2}
    {\Gamma \types \fPush (x.f_1,x.f_2) \type \\\\
    \Gamma,y_1:B_1\types \nu_1(y_1):\fPush (x.f_1,x.f_2) \\
    \Gamma,y_2:B_2\types \nu_2(y_2):\fPush (x.f_1,x.f_2) \\
    \Gamma,a:A\types \mu(a):\id[\fPush(x.f_1,x.f_2)]{\nu_1(f_1[a/x])}{\nu_2(f_2[a/x])}}\and
  % 
    \inferrule{\Gamma\types A\type \\ \Gamma\types B_1\type \\ \Gamma\types B_2\type \\\\
      \Gamma,x:A\types f_1:B_1 \\ \Gamma,x:A \types f_2:B_2 \\\\ \Gamma, u:\fPush(x.f_1,x.f_2) \types C\type \\
      \Gamma, y_1:B_1\types t_1:C[\nu_1(y_1)/u] \\ \Gamma, y_2:B_2\types t_2:C[\nu_2(y_2)/u] \\
      \Gamma, x:A \types m:\idover[C]{t_1[f_1(a)/y_1]}{{t_2[f_2(a)/y_2]}}{\mu(x)}}
    {\Gamma,p:\fPush(x.f_1,x.f_2)\types \mathsf{pe}(u.C,y_1.t_1,y_2.t_2,a.m,p) : C[p/u] \\
    \Gamma,b_1:B_1 \types \mathsf{pe}(u.C,y_1.t_1,y_2.t_2,a.m,\nu_1(b_1)) \jdeq t_1[b_1/y_1] \\
    \Gamma,b_2:B_2 \types \mathsf{pe}(u.C,y_1.t_1,y_2.t_2,a.m,\nu_2(b_2)) \jdeq t_2[b_2/y_2] \\
    \Gamma,a:A \types \ap_{u.\mathsf{pe}(u.C,y_1.t_1,y_2.t_2,a.m,u)}(a) \jdeq m[a/x]}\and
  \end{mathpar}
  \caption{Pushouts}
  \label{fig:pushouts}
\end{figure}

We now briefly sketch the local universe coherence theorems for all of these structures; they are all straightforward applications of the techniques of~\cite{lw:localuniv}.

\begin{thm}
  If $\C$ has a stable class of identity types and satisfies~\eqref{eq:lf}, then $\C_!$ has strictly stable identity types.
\end{thm}
\begin{proof}
  Just like~\cite[Theorem 3.4.3.2]{lw:localuniv}, only simpler due to the absence of Frobenius.
  If $A\in\T_!(\Gamma)$ are represented by $E_A\in\T(V_A)$, then $\Id_A\in\T_!(\Gamma.A.A)$ is represented by a chosen good identity type $\Id_{E_A}\in \T(V_A.E_A.E_A)$, with reflexivity term also chosen in the same universal case.
  The local universe for the elimination and computation rules is just as in~\cite[Theorem 3.4.3.2]{lw:localuniv} without the types ``$B_i$'':
  \begin{align*}
    V = [ & a:V_A, \\
    & c:\prod x,x':E_A(a), y:\Id_{E_A}(a,x,x'). V_C,\\
    & d:\prod x:E_A(a). E_C(c(x,x,\r_{E_A}(a,x)))].
  \end{align*}
  (As in~\cite{lw:localuniv}, we express local universes as iterated $\Sigma$-types in the internal extensional type theory of $\C$.)
\end{proof}

\begin{thm}
  If $\C$ has a stable class of dependent identity types and satisfies~\eqref{eq:lf}, then $\C_!$ has strictly stable dependent identity types.
\end{thm}
\begin{proof}
  If $A\in\T_!(\Gamma)$ and $B\in\T_!(\Gamma.A)$ is represented by $E_A\in\T(V_A)$ and $E_B\in\T(V_B)$, then $\Id^A_B\in\T_!(\Gamma.A.A.\Id_A.B.B)$ is represented by a chosen good dependent identity type for $E_B$ over $E_A$, with the local universe
  \begin{align*}
    [ a:V_A, b: (V_B)^{E_A(a)} ]
    % & x:E_A(a), x':E_A(a), y:\Id_A(a,x,x'), \\
    % &u:E_B(b(x)), u':E_B(b(x')) ]
  \end{align*}
  with dependent reflexivity term likewise chosen in this case.
  (Note that this local universe is the same one $V_A \triangleleft V_B$ used in~\cite{lw:localuniv} for $\Pi$-types and $\Sigma$-types.)
  The local universe for the elimination and computation rules is
  \begin{align*}
    [ & a:V_A, \\
    & b:\prod x:E_A(a). V_B,\\
    & c:\prod x,x':E_A(a), y:\Id_{E_A}(a,x,x'), u:E_B(b(x)), u':E_B(b(x')), z:\Id^{E_A}_{E_B}(a,x,x',y,b,u,u'). V_C,\\
    & d:\prod x:E_A(a), u:E_B(b(x)) . E_C(c(x,x,\r_{E_A}(a,x),u,u,\r^{E_A}_{E_B}(a,x,u)))].\qedhere
  \end{align*}
\end{proof}

It is perhaps worth emphasizing how the above local universes are obtained essentially algorithmically from the rules in \cref{fig:depid}, once given that the latter are written in the style of \cref{rmk:rule-style}.
Namely, each premise of a rule corresponds to one term in the iterated $\Sigma$-type, which is a $\Pi$-type of the consequent of that premise over the additional variables (beyond the arbitrary context $\Gamma$) in the context of that premise.

\begin{thm}
  If $\C$ has a stable class of identity applications and satisfies~\eqref{eq:lf}, then $\C_!$ has strictly stable identity applications.
\end{thm}
\begin{proof}
  The local universe is
  \begin{align*}
    [ & a:V_A \\
    & b:\prod x:E_A(a). V_B,\\
    & f:\prod x:E_A(a). E_B(b(x)) ].\qedhere
  \end{align*}
\end{proof}

\begin{thm}\label{thm:lu-pushout}
  If $\C$ has a stable class of typal pushouts and satisfies~\eqref{eq:lf}, then $\C_!$ has strictly stable typal pushouts.
\end{thm}
\begin{proof}
  The local universe for the formation and introduction rules is
  \begin{align*}
    [ & a:V_A, b_1:V_{B_1}, b_2:V_{B_2}, \\
    &f_1: \prod x:E_A(a). E_{B_1}(b_1),\\
    &f_2: \prod x:E_A(a). E_{B_2}(b_2) ].
  \end{align*}
  And the local universe for the elimination and computation rules is
  \begin{align*}
    [ & a:V_A, b_1:V_{B_1}, b_2:V_{B_2}, \\
    &f_1: \prod x:E_A(a). E_{B_1}(b_1),\\
    &f_2: \prod x:E_A(a). E_{B_2}(b_2),\\
    &c : \prod u:E_{\fPush(f_1,f_2)}(a,b_1,b_2,f_1,f_2). V_C, \\
    &t_1 : \prod y_1:E_{B_1}(b_1). E_C(c(\nu_1(y_1))), \\
    &t_2 : \prod y_2:E_{B_2}(b_2). E_C(c(\nu_2(y_2))), \\
    &m : \prod x:E_A(a). \Id^{\fPush(f_1,f_2)}_C(a,b_1,b_2,f_1,f_2,c,\nu_1(f_1(x)),\nu_2(f_2(x)),\mu(x),t_1(f_1(x)),t_2(f_2(x)))
    ].%\qedhere
  \end{align*}
\end{proof}

\begin{thm}
  For any good model category $\sM$, the split comprehension category $\fibmbang$ has strictly stable typal pushouts, relative to the strictly stable identity types, dependent identity types, and identity applications obtained by strictifying its canonical weakly stable ones.
\end{thm}
\begin{proof}
  Combine \cref{thm:model-pushout,thm:lu-pushout}.
\end{proof}

This is almost, but not quite, the result we want.
The pushout rules in \cref{fig:pushouts} differ from those commonly used (e.g.\ in~\cite{hottbook}) by the presence of the additional rules in \cref{fig:depid,fig:ap}.
The usual approach is instead to start with only identity types as in \cref{fig:id}, and then \emph{define} dependent identity types using one of the methods~\ref{item:idover1}--\ref{item:idover3} listed on page~\pageref{idover}, similarly define $\ap$ using the eliminator of identity types, and then state rules such as those in \ref{fig:pushouts} relative to these defined structures.

Such definitions certainly give \emph{some} classes of dependent identity types and identity applications, which are strictly stable if the identity types we started with are strictly stable.
However, if such constructions are performed in $\fibmbang$, the resulting strictly stable classes will not generally be the \emph{same} as those obtained from the canonical weakly stable ones by the local universes construction.
They will indeed be \emph{equivalent}, and we can transfer the structure of a pushout across such an equivalence, but at the cost of weakening the computation rule for the path-constructor from a judgmental equality to the existence of a path, as follows.

\begin{defn}\label{defn:wk-pushouts}
  A \textbf{stable class of weak typal pushouts} in a comprehension category $(\C,\T)$, relative to given stable classes of identity types, dependent identity types, and identity applications, consists of all the data and properties in \cref{defn:pushout} except that instead of $\ap_s\circ \mu = m$, we have for any good identity type $\Id_{\Id^D_C}\in\T(\Gamma.D.D.\Id_D.C.C.\Id^D_C.\Id^D_C)$ a non-empty family of ``good'' maps $v:A\to \Id_{\Id^D_C}$ over $(\ap_s\circ \mu, m)$.
  If $\C$ is split and each family of good structures is a singleton, we say that $\C$ has \textbf{strictly stable weak typal pushouts}.
\end{defn}

Of course, typal pushouts are also weak typal pushouts, since if $\ap_s\circ \mu = m$ we can compose this map with $\r$ to get $v$.
The corresponding modified elimination rule in syntactic type theory is shown in~\cref{fig:wk-pushout}.
In the case when the dependent identity types and identity applications are defined from the ordinary identity types as on page~\pageref{idover}, this yields exactly the notion of pushout type used in~\cite[\S6.8]{hottbook}.
Some reasons for choosing a weak computation rule for the path-constructor are adumbrated in~\cite[\S6.2 and Chapter 6 Notes]{hottbook}, one of which (non-strictness of the equivalence between left and right homotopies) is roughly the same reason such a rule appears here.

\begin{figure}
  \centering
  \begin{mathpar}
    \inferrule{\Gamma\types A\type \\ \Gamma\types B_1\type \\ \Gamma\types B_2\type \\\\
      \Gamma,x:A\types f_1:B_1 \\ \Gamma,x:A \types f_2:B_2 \\\\ \Gamma, u:\fPush(x.f_1,x.f_2) \types C\type \\
      \Gamma, y_1:B_1\types t_1:C[\nu_1(y_1)/u] \\ \Gamma, y_2:B_2\types t_2:C[\nu_2(y_2)/u] \\
      \Gamma, x:A \types m:\idover[C]{t_1[f_1(a)/y_1]}{{t_2[f_2(a)/y_2]}}{\mu(x)}}
    {\Gamma,a:A \types \mathsf{pec}(u.C,y_1.t_1,y_2.t_2,a.m,a) : \id{\ap_{u.\mathsf{pe}(u.C,y_1.t_1,y_2.t_2,a.m,u)}(a)}{m[a/x]}}\and
  \end{mathpar}
  \caption{Computation for weak pushouts}
  \label{fig:wk-pushout}
\end{figure}

The following theorem could be stated for weakly stable structures in addition to strict ones, but we only need to apply it after the local universes splitting has occurred, so we don't bother with that generality.

\begin{thm}
  Let $\C$ be a split comprehension category equipped with strictly stable identity types.
  If $\C$ has strictly stable weak typal pushouts with respect to one choice of strictly stable dependent identity types and identity applications, then with respect to any other choice of strictly stable dependent identity types and identity applications it also has strictly stable weak typal pushouts.
\end{thm}
\begin{proof}
  We use the ordinary notation for the given identity types, dependent identity types, identity applications, and typal pushouts, and we write $\idovertwo[B]{x}{y}{e}$ and $\aptwo{f}$ and so on for some other choice of dependent identity types and identity applications.
  Using the eliminators for dependent identity types in both directions (or equivalently the uniqueness-up-to-homotopy of weak factorization systems), we obtain maps in both directions making all the triangles commute:
  \[
  \begin{tikzcd}
    & \Gamma.A.A.\Id_A.B.B.\Id^A_B \ar[dr] \ar[dd,shift left,"h"]\\
    \Gamma.A.B \ar[ur,"{\r^A_B}"] \ar[dr,swap,"{\rtwo{A}{B}}"] && \Gamma.A.A.\Id_A.B.B\\
    & \Gamma.A.A.\Id_A.B.B.\Idtwo{A}{B} \ar[ur] \ar[uu,shift left,"k"]
  \end{tikzcd}
  \]
  Similarly, we obtain homotopies over $\Gamma.A.A.\Id_A.B.B$ between the round-trip composites in both directions (i.e.\ maps $\Gamma.A.B \to \Gamma.A.A.\Id_A.B.B.\Id^A_B.\Id^A_B.\Id_{\Id^A_B}$ and similarly), giving a homotopy equivalence $\Id^A_B \simeq \Idtwo{A}{B}$.
  And if we have a section $f$ of $B\in\T(\Gamma.A)$, then the composite
  \[ \Gamma.A.A.\Id_A \xto{\ap_f} \Gamma.A.A.\Id_A.B.B.\Id^A_B \xto{h} \Gamma.A.A.\Id_A.B.B.\Idtwo{A}{B} \]
  is a lift of $(f,f)$, and has the property that when composed with $\r_A : \Gamma.A \to \Gamma.A.A.\Id_A$ it becomes $\rtwo{A}{B}$.
  Since $\aptwo{f}$ also has these properties, we can use identity elimination to produce a homotopy $h\circ \ap_f \sim \aptwo{f}$ over $(f,f)$, i.e.\ a lift of $(h\circ \ap_f,\aptwo{f})$ to $\Gamma.A.A.\Id_A.B.B.\Idtwo{A}{B}.\Idtwo{A}{B}.\Id_{\Idtwo{A}{B}}$.

  Now, note that only the eliminator and computation rules for pushouts refer to dependent identity types; the formation and introduction rules can be exactly as given.
  For the eliminator, all the structure on $C$ in the second case can be used as-given except for $u:A\to \Gamma.D.D.\Id_D.C.C.\Idtwo{D}{C}$, which we compose with $k$ to produce data for the given eliminator.
  This produces a section of the desired form, satisfying the computation rules $s\circ \nu_i = t_i$ and with a homotopy $\ap_s\circ \mu \sim k\circ m$.
  Postcomposing this homotopy with $h$ (using ordinary $\Id$-elimination) and concatenating with the above-constructed homotopies $h\circ \ap_s \sim \aptwo{s}$ and $h\circ k \sim 1$, we get
  \[ \aptwo{s} \circ \mu \sim h\circ \ap_s \circ \mu \sim h\circ k\circ m \sim m \]
  which is what we want.
  Since everything that went into the construction was assumed strictly stable, so is the result.
\end{proof}

\begin{cor}
  For any good model category $\sM$, the split comprehension category $\fibmbang$ has strictly stable weak typal pushouts, relative to the strictly stable identity types obtained by strictifying its canonical weakly stable ones, and the dependent identity types obtained from any of~\ref{item:idover1}--\ref{item:idover3} on page~\pageref{idover} and the identity applications obtained from $\Id$-elimination.\qed
\end{cor}

\section{Natural numbers}
\label{sec:natural-numbers}

The natural numbers type, though a standard type constructor, is not considered explicitly in~\cite{lw:localuniv}; thus we begin with its local universes theory.

\begin{defn}
  In a comprehension category $(\C,\T)$, a \textbf{natural numbers type} over $\Gamma\in\C$ is a type $\N\in\T(\Gamma)$ together with a section $\zero:\Gamma\to\Gamma.\N$ and a map $\succ:\Gamma.\N\to\Gamma.\N$ over $\Gamma$, plus an operation assigning to any type $C\in\T(\Gamma.\N)$ equipped with $z:\Gamma\to\Gamma.\N.C$ over $\zero$ and $s:\Gamma.\N.C \to \Gamma.\N.\C$ over $\succ$, a section $f$ of $C$ such that $f \circ \zero = z$ and $f\circ \succ = s \circ f$.

  We say $\C$ has a \textbf{weakly stable natural numbers type} if for any $\Gamma$ there is a natural numbers type $\N\in\T(\Gamma)$ such that for any $\sigma:\Delta\to\Gamma$, $(\N[\sigma],\zero[\sigma],\succ[\sigma])$ is a natural numbers type over $\Delta$.
  Similarly, if $\C$ is split, it has a \textbf{strictly stable natural numbers type} if it has an operation assigning to each $\Gamma$ a natural numbers type $\N\in\T(\Gamma)$ such that reindexing along any $\sigma:\Delta\to\Gamma$ preserves natural numbers types along with their data and specified sections.
\end{defn}

Of course, if $\C$ has a terminal object (as it usually does), then the quantification over $\Gamma$ in the latter definition follows from the special case $\Gamma=1$.
A strictly stable natural numbers type is the semantic version of the usual rules for a natural numbers type, shown in \cref{fig:nno}.

\begin{figure}
  \centering
  \begin{mathpar}
    \inferrule{ }{\Gamma\types\N\type}\and
    \inferrule{ }{\Gamma\types \zero:\N}\and
    \inferrule{ }{\Gamma,x:\N\types \succ(x):\N}\and
    \inferrule{\Gamma,x:\N\types C\type \\ \Gamma\types z:C[\zero/x] \\ \Gamma,x:\N,y:C \types s : C[\succ(x)/x]}{\Gamma,n:\N\types \nrec(x.C,z,xy.s,n) : C[n/x] \\ \Gamma\types \nrec(x.C,z,xy.s,\zero) \jdeq z\\ \Gamma,n:\N \types \nrec(x.C,z,xy.s,\succ(n)) \jdeq s[n/x,\nrec(x.C,z,xy.s,n)/y]}
  \end{mathpar}
  \caption{Natural numbers type}
  \label{fig:nno}
\end{figure}

\begin{thm}
  If $\C$ satisfies~\eqref{eq:lf} and has a weakly stable natural numbers type, then $\C_!$ has a strictly stable natural numbers type.
\end{thm}
\begin{proof}
  Note that condition~\eqref{eq:lf} includes the existence of a terminal object.
  We take the terminal object to be the local universe for the formation and introduction rules, choosing a weakly stable natural numbers type over it.
  For the elimination and computation rules, the local universe is
  \begin{align*}
    [&c: \prod x:E_{\N}. V_C,\\
    &z: E_C(c(\zero)),\\
    &s: \prod x:E_{\N}, y:E_C(c(x)) . E_C(c(\succ(x))) ].\qedhere
  \end{align*}
\end{proof}

In a good model category, the obvious candidate for a natural numbers type is the countable coproduct $\coprod_{n:\dN} 1$ of copies of the terminal object, which is automatically a ``natural numbers object'' in the usual sense of category theory.
As usual, the problem is that in general this may not be fibrant (though it is in many examples, such as simplicial sets).
But unlike in the preceding sections, it does not suffice to simply fibrantly replace it.
If $\N$ is a fibrant replacement of $\coprod_{n:\dN} 1$, we can extend the zero and successor operations to it, and for any fibration $C\fib\N$ equipped with $z$ and $s$ we can find a section of $C$ over $\coprod_{n:\dN} 1$ using its universal property and then extend that section to $\N$ by lifting against the acyclic cofibration $\coprod_{n:\dN} 1 \to \N$.

However, such a lift will not generally satisfy the computation rules.
It might be possible to choose it cleverly to satisfy the computation rule for $\zero$, but for $\succ$ there is essentially no hope, since the computation rule for $\succ$ relates the section to \emph{itself}, which is impossible to obtain using a simple lifting property.
Thus, we have to be more clever.

Let $F_\N$ denote the endofunctor of $\sM$ defined by $F_\N(X) = X+1$.
Recall that an \textbf{endofunctor-algebra} for $F_\N$ is an object $X$ together with a map $F_\N(X) \to X$.
A natural numbers object in the ordinary sense is precisely an \emph{initial} such endofunctor-algebra.
And a natural numbers type over the terminal object is precisely a fibrant $F_\N$-algebra $\N$ such that any fibration $C\to\N$ that is also a map of $F_\N$-algebras has an $F_\N$-algebra section.
This suggests that $\N$ is an ``initial fibrant $F_\N$-algebra''; we now make this precise using the technology of \emph{algebraic weak factorization systems}~\cite{gt:nwfs,garner:soa,riehl:nwfs-model}.
For this we strengthen our standing assumptions as follows.

\begin{defn}
  An \textbf{excellent model category} $\sM$ is a good model category that is in addition \emph{combinatorial}, i.e.\ a locally presentable category and cofibrantly generated as a model category.
\end{defn}

This allows us to use the technology of algebraically-free monads and algebraic colimits of monads~\cite{kelly:transfinite,nlab:transfinite}.
Let $\dT_\N$ denote the algebraically-free monad on $F_\N$, which exists since $F_\N$ is an accessible endofunctor.
This is means it is a monad whose monad-algebras are precisely the endofunctor-algebras for $F_\N$.
By construction of $\dT_\N$, it is also accessible.

Now, by the construction of~\cite{garner:soa} and the combinatorial-ness of $\sM$, we have an accessible \textbf{fibrant factorization monad} $\dR$ on the arrow category $\sM^\to$, such that an arrow can be given an $\dR$-algebra structure if and only if it is a fibration, and $\dR$ preserves codomains.
In particular, when restricted to arrows with terminal codomain, we get an accessible \textbf{fibrant replacement monad} $\dR_1$ on $\sM$ itself, such that an object can be given an $\dR_1$-algebra structure if and only if it is fibrant.
Moreover, the composite of two $\dR$-algebras $X \xfib{f} Y \xfib{g} Z$ naturally acquires an $\dR$-algebra structure such that the commutative square
\[
\begin{tikzcd}
  X \ar[d,"gf",swap] \ar[r,"f"] & Y \ar[d,"g"] \\ Z \ar[r,idmap]& Z
\end{tikzcd}
\]
is a morphism of $\dR$-algebras.

\begin{thm}
  If $\sM$ is an excellent model category, then $\fibm$ has a natural numbers type over $1$.
\end{thm}
\begin{proof}
  Consider the algebraic coproduct of monads $\dT_\N+\dR_1$.
  This is an (accessible) monad such that a $(\dT_\N+\dR_1)$-algebra structure on an object $X$ is precisely a pair of an (unrelated) $\dT_\N$-algebra structure (i.e.\ an $F_\N$-endofunctor-algebra structure) and an $\dR_1$-algebra structure.
  Let $\N$ be the initial $(\dT_\N+\dR_1)$-algebra, i.e.\ the free $(\dT_\N+\dR_1)$-algebra on the initial object, $\N =(\dT_\N+\dR_1)(\emptyset)$.
  Since $\N$ is a $\dT_\N$-algebra, it comes with $\zero$ and $\succ$; and since it is an $\dR_1$-algebra, it is fibrant.

  Now suppose $p:\N.C\fib \N$ is a fibration of $F_\N$-endofunctor-algebras.
  Choose an $\dR$-algebra structure on $p$.
  Then the composite $\N.C\fib\N\fib 1$ acquires an $\dR$-algebra structure, which is to say that $\N.C$ acquires an $\dR_1$-algebra structure, such that $p$ is (not just an $\dR$-algebra but) an $\dR_1$-algebra map.
  Since $\N.C$ is also a $\dT_\N$-algebra and $p$ is also a $\dT_\N$-algebra map, it follows that $\N.C$ is a $(\dT_\N+\dR_1)$-algebra and $p$ is a $(\dT_\N+\dR_1)$-algebra map.
  But since $\N$ is the initial $(\dT_\N+\dR_1)$-algebra, any $(\dT_\N+\dR_1)$-algebra map with codomain $\N$ has a $(\dT_\N+\dR_1)$-algebra section, and hence in particular a $\dT_\N$-algebra section, as desired.
\end{proof}

In general it does not seem possible to prove that this natural numbers type is weakly stable, but with a small modification  we can.
Let $\Mf$ denote the subcategory of fibrant objects in $\sM$, and $\fibmf$ the restriction of the comprehension category $\fibm$ to $\Mf$.
Note that $\Mf$ is the smallest subcategory of $\sM$ to which $\cF$ can be restricted and still be a comprehension category, since every fibrant object is the comprehension of some fibration over $1$.
In particular, any semantics of type theory in $\fibm$ must land inside $\Mf$, so nothing important is lost by this restriction.

\begin{lem}
  For any good model category $\sM$, the comprehension category $\fibmf$ also satisfies~\eqref{eq:lf}.
\end{lem}
\begin{proof}
  The product of fibrant objects is of course fibrant, and the assumptions on a good model category ensure that the dependent exponential of a fibration along a fibration is again a fibration, hence its domain is fibrant if its codomain is.
\end{proof}

Moreover, any weakly stable structure possesed by $\fibm$ is also possesed by $\fibmf$.
Thus, all our preceding theorems about strictly stable structure in $\fibmbang$ also apply to $\fibmfbang$.

\begin{thm}\label{thm:nat-stable}
  If $\sM$ is an excellent model category, then $\fibmf$ has a weakly stable natural numbers type.
\end{thm}
\begin{proof}
  It suffices to show that any natural numbers type $\N$ over $1$ is weakly stable, which is to say that for any (fibrant!)\ object $\Gamma$ the pullback $\Gamma^* \N$ is a natural numbers type over $\Gamma$.
  The proof is essentially the usual category-theoretic argument that pullback functors preserve natural numbers objects.
  Suppose $C\fib \Gamma^* \N$ is a fibration over $\Gamma$ equipped with a section $z:\Gamma\to C$ over $\Gamma^*(\zero)$ and a map $s:C\to C$ over $\Gamma^*(\succ)$.
  Then (since $\Gamma$ is fibrant) $\Gamma_* C \fib \Gamma_* \Gamma^* \N$ is a fibration, equipped with a point $\Gamma_*(z) : \Gamma_*(\Gamma^*(1)) = \Gamma_*(\Gamma) \to \Gamma_*(C)$ over $\Gamma_*(\Gamma^*(\zero))$ and a map $\Gamma_*(s):\Gamma_*(C)\to\Gamma_*(C)$ over $\Gamma_*(\Gamma^*(\succ))$.

  Now pull $\Gamma_* C$ back to $\N$ along the unit $\eta_\N$ of the adjunction $\Gamma^* \adj \Gamma_*$.
  Since $\eta$ is natural we have $\Gamma_*(\Gamma^*(\zero)) \circ \eta_1 = \eta_\N \circ \zero$ and $\Gamma_*(\Gamma^*(\succ)) \circ \eta_\N = \eta_\N \circ \succ$, so we have an induced map $z' : 1 \to \eta_\N^*\Gamma_*C$ over $\zero$ and $s':\eta_\N^*\Gamma_*C \to \eta_\N^*\Gamma_*C$ over $\succ$.
  Thus, using $\nrec$ for $\N$ we have a section $f:\N\to \eta_\N^*\Gamma_* C$ such that $f \circ \zero = z'$ and $f\circ \succ = s' \circ f$, or equivalently a map $g:\N \to \Gamma_*C$ over $\eta_\N$ such that $g\circ \zero = \Gamma_*(z) \circ \eta_1$ and $g\circ \succ = \Gamma_*(s) \circ g$.
  Finally, transposing $g$ across the adjunction $\Gamma^* \adj \Gamma_*$, we obtain $\Gamma^* \N \to C$ over $1_\N$ (the adjunct of $\eta_\N$) such that $h\circ \Gamma^*(\zero) = z$ and $h\circ \Gamma^*(\succ) = s\circ h$, as desired.
\end{proof}

\begin{cor}
  If $\sM$ is an excellent model category, then $\fibmfbang$ has a strictly stable natural numbers type.\qed
\end{cor}


\section{\W-types}
\label{sec:w-types}

The construction of \W-types is similar to that of natural numbers, with one additional wrinkle.
Namely, the presence of parameters means that in weak stability we have to consider pullback along arbitrary morphisms $\sigma:\Delta\to\Gamma$ between fibrant objects, and in the case when $\Gamma$ is not terminal, fibrancy of $\Delta$ does not imply that $\sigma$ is a fibration.

Since a good model category is locally cartesian closed, it has \emph{pseudo-stable $\Pi$-types}~\cite[Definition 3.4.2.8]{lw:localuniv}, so we can use the definition of \textbf{weakly stable \W-types} relative to these from~\cite[Definition 3.4.4.9]{lw:localuniv}.

\begin{thm}
  If $\sM$ is an excellent model category, then $\fibm$ has \W-types.
  That is, for any $\Gamma\in\sM$, $A\in\F(\Gamma)$, and $B\in\F(\Gamma.A)$, there exists a \W-type as in~\cite[Definition 3.4.4.7]{lw:localuniv}.
\end{thm}
\begin{proof}
  Let $F_{\W A B}$ be the endofunctor of $\sM/\Gamma$ sending $X \to \Gamma$ to the local exponential $(A^*X)^B_A$, equipped with the composite projection $(A^*X)^B_A \to A\to \Gamma$.
  Since this is a composite of left or right adjoints between accessible categories, it is an accessible endofunctor, and $\sM/\Gamma$ is locally presentable, so $F_{\W A B}$ generates an algebraically-free monad $\dT_{\W A B}$.
  Let $\Gamma.\W$ be the initial algebra of $\dT_{\W A B} + \dR_{\Gamma}$, where $\dR_\Gamma$ is the restriction of $\dR$ to morphisms with target $\Gamma$.
  Then $\Gamma.\W\to \Gamma$ is a fibration, since it is an $\dR$-algebra, and its $F_{\W A B}$-endofunctor-algebra structure is exactly the map $\fold$ required of a \W-type.

  The input to the eliminator consists, as usual, of a fibration $p:\Gamma.\W.C\fib \Gamma.W$ that is a $F_{\W A B}$-algebra morphism.
  This is not completely obvious as stated in~\cite[Definition 3.4.4.7]{lw:localuniv}; the point is that $\Gamma.A.\Pi[B,\W].\Pi[B,C[\mathsf{app}'_{B,W}]] \cong \Gamma.A.\Pi[B,\Sigma[\W,C]]$ by the mapping-in universal property of $\Sigma$-types; and by construction of $\Sigma$-types in $\sM$, the comprehension $\Gamma.\Sigma[\W,C]\fib\Gamma$ is just the composite $\Gamma.\W.C \fib \Gamma.W\fib \Gamma$.\footnote{Also, I think there is a typo in~\cite[Definition 3.4.4.7]{lw:localuniv}; the $\Gamma.\W.\dots$ in the diagrams should be $\Gamma.A.\dots$.})
  As before, we choose an $\dR$-algebra structure on $p$, making this composite $\Gamma.\W.C \fib \Gamma.W\fib \Gamma$ an $\dR$-algebra and the square
  \[
  \begin{tikzcd}
    \Gamma.\W.C \ar[d,->>] \ar[r,->>] & \Gamma.\W \ar[d,->>] \\ \Gamma \ar[r,idmap]& \Gamma
  \end{tikzcd}
  \]
  an $\dR$-algebra map, hence also a $(\dT_{\W A B} + \dR_{\Gamma})$-algebra map.
  Thus, since $\W$ is the initial $(\dT_{\W A B} + \dR_{\Gamma})$-algebra, this morphism has a $(\dT_{\W A B} + \dR_{\Gamma})$-algebra section, and in particular a $F_{\W A B}$-endofunctor-algebra section, which is $\wrec$.
\end{proof}

\begin{thm}
  If $\sM$ is an excellent model category, then $\fibmf$ has weakly stable \W-types.
\end{thm}
\begin{proof}
  Let $\W\fib\Gamma$ be a $\W$-type constructed as above for $\Gamma\in\sM$, $A\in\F(\Gamma)$, and $B\in\F(\Gamma.A)$; we must show that its pullback along any $\sigma:\Delta\to\Gamma$ is also a \W-type.
  We consider separately the cases when $\sigma$ is an acyclic cofibration or a fibration; by factorization this suffices for the general case.

  If $\sigma$ is a fibration, the argument is similar to that of \cref{thm:nat-stable}.
  Suppose given a fibration $C\fib \sigma^*\W$ over $\Delta$ with structure map $d$ over $\sigma^*(\fold)$:
  \[
  \begin{tikzcd}
    (\sigma^* A)_! (\sigma^* f)_* (\sigma^*B)^* C \ar[rr,"d"] \ar[d] && C \ar[d] \\
    (\sigma^* A)_! (\sigma^* f)_* (\sigma^*B)^* \sigma^* \W \ar[r,"\cong"] & \sigma^*(f_* B^* \W) \ar[r,"{\sigma^*(\fold)}",swap] & \sigma^*\W
  \end{tikzcd}
  \]
  Then using several Beck-Chevalley transformations, we have a composite map
  \begin{multline*}
    A_! f_* B^* \sigma_* C
    \toiso A_! f_* (B^*\sigma)_* (\sigma^*B)^* C
    \toiso A_! (A^*\sigma)_* (\sigma^* f)_* (\sigma^*B)^* C\\
    \to \sigma_* (\sigma^* A)_! (\sigma^* f)_* (\sigma^*B)^* C
    \xto{\sigma_* d} \sigma_* C
  \end{multline*}
  that lies over the corresponding map for $\sigma^*\W$ constructed from $\sigma^*(\fold)$.
  Now if we pull this back along the unit $\eta_\W : \W \to \sigma_* \sigma^* \W$ we get a map $d'$ satisfying
  \[
  \begin{tikzcd}
    A_! f_* B^* \eta_\W^* \sigma_* C \ar[r,"d'"] \ar[d] & \eta_\W^* \sigma_* C \ar[d] \\
    A_! f_* B^* \W \ar[r,"\fold"] & \W
  \end{tikzcd}
  \]
  and thus a section $\wrec:\W\to \eta_\W^* \sigma_* C$ commuting with $d'$ and $\fold$.
  Transposing this back across the adjunction $\sigma^* \adj \sigma_*$, we obtain a section of $C$ commuting with $d$ and $\sigma^*(\fold)$, as desired.

  On the other hand, if $\sigma$ is an acyclic cofibration, then so is its pullback $\W^*\sigma : \sigma^*\W \to \W$ along the fibration $\W\fib\Gamma$.
  Moreover, since $\Delta$ is fibrant, so is the object $\sigma^*\W$.
  Therefore, the acyclic cofibration $\W^*\sigma$ admits a retraction $r$.
  Since $(\W^*\sigma)^* r^* C \cong C$, it suffices to extend the given structure $d$ on the fibration $C\fib\sigma^*\W$ to a corresponding structure on $r^* C\fib\W$ whose pullback to $\sigma^*\W$ is $d$.
  This is contained in the following lemma.
\end{proof}

\begin{lem}
  Suppose a pair of pullback squares in a good model category $\sM$:
  \[
  \begin{tikzcd}
    D \ar[r,>->,"i"] \ar[d,->>,"p",swap] \ar[dr,phantom,near start,"\lrcorner"] & C \ar[d,->>,"q"] \\
    V \ar[r,>->,"j"] \ar[d,->>] \ar[dr,phantom,near start,"\lrcorner"] & W \ar[d,->>] \\
    \Delta \ar[r,>->,"\sigma",swap] & \Gamma
  \end{tikzcd}
  \]
  in which all objects are fibrant, the downward-pointing arrows are fibrations, and the rightward-pointing arrows are acyclic cofibrations.
  Suppose moreover that for some fibration $f:B\to A$ between fibrant objects of $\sM/\Gamma$, the object $W$ of $\sM/\Gamma$ is a $F_{\W A B}$-endofunctor-algebra, the objects $D$ is a $F_{\W (\sigma^*A) (\sigma^*B)}$-endofunctor-algebra, and $p$ is a $F_{\W (\sigma^*A) (\sigma^*B)}$-morphism (when $V$ has its induced structure from $W$).
  Then there is a $F_{\W A B}$-endofunctor-algebra structure on $C$ inducing the structure on $D$ by pullback and making $q$ a $F_{\W A B}$-morphism.
\end{lem}
\begin{proof}
  By definition, $F_{\W A B}(X) = f_* B^* X$ (with its canonical map to $A$ forgotten), and similarly $F_{\W (\sigma^*A) (\sigma^*B)}(X) = (\sigma^*f)_* (\sigma^*B)^* X$.
  Thus we are given the following commutative diagram of solid arrows and we want to construct the dashed arrow making the other squares commute:
  \[
  \begin{tikzcd}
    (\sigma^*f)_* (\sigma^*B)^* D \ar[rr] \ar[dr] \ar[dd] && f_* B^* C \ar[dr,dashed] \ar[dd]\\
    & D \ar[rr,>->,near end,"i",swap,crossing over] && C \ar[dd,->>,"q"] \\
    (\sigma^*f)_* (\sigma^*B)^* V \ar[rr] \ar[dr] && f_* B^* W \ar[dr] \\
    & V \ar[rr,>->,"j",swap] \ar[from=uu,crossing over,->>,near end,"p"] && W
  \end{tikzcd}
  \]
  This is equivalent to solving the following lifting problem:
  \[
  \begin{tikzcd}
    (\sigma^*f)_* (\sigma^*B)^* D \ar[r] \ar[d] & D \ar[r] & C \ar[d,->>] \\
    f_* B^* C \ar[urr,dashed] \ar[r] & f_* B^* W \ar[r] & W
  \end{tikzcd}
  \]
  so it suffices to show that $(\sigma^*f)_* (\sigma^*B)^* D \to f_* B^* C$ is an acyclic cofibration.
  Now since $D = \sigma^*C$, we have $(\sigma^*B)^*D = \sigma^* B^* C$, so the map $(\sigma^*B)^*D \to B^* C$ is a pullback of $\sigma$ along the composite fibration $B^*C \to C \to \Gamma$, hence an acyclic cofibration.
  But $(\sigma^*B)^*D$ is also the pullback of $B^* C$ along the map $B^*\sigma : \sigma^*B \to B$, so by the Beck-Chevalley condition for the pullback square
  \[
  \begin{tikzcd}
    \sigma^* B \ar[r,"{B^*\sigma}"] \ar[d,"{\sigma^* f}",swap] & B \ar[d,"f"]\\
    \sigma^* A \ar[r,"{A^*\sigma}",swap] & A
  \end{tikzcd}
  \]
  we have $(\sigma^*f)_* (\sigma^*B)^* D \cong (A^*\sigma)^* f_* B^* C$.
  Thus, our desired map $(\sigma^*B)^*D \to B^* C$ is equivalently the pullback $(A^*\sigma)^* f_* B^* C \to f_* B^* C$ of the acyclic cofibration $A^*\sigma$ along the fibration $f_* B^* C \fib A$, and hence an acyclic cofibration.
\end{proof}

\begin{cor}
  If $\sM$ is an excellent model category, then $\fibmfbang$ has strictly stable \W-types.
\end{cor}


\bibliographystyle{alpha}
\bibliography{basictex/all}

\end{document}
